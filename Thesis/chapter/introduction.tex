
\chapter{Introduction}

\todo{Ignore this chapter. I just copied this from the paper.}
Prior work has established a Curry-Howard correspondence between intuitionistic linear sequent calculus and session-typed message-passing concurrency \cite{CairesP10, PfenningG15, Honda93}. In this formulation, linear propositions are interpreted as session-types, proofs as processes, and cut elimination as communication. Session types are assigned to channels and prescribe the communication behavior along them. Each channel is offered by a unique process and used by exactly one, which is ensured by linearity. When the behavior along a channel $c$ satisfies the type $A$ and $P$ is the process that offers along $c$, we say that $P$ provides a session of type $A$ along $c$.

In the base system, each type directly corresponds to a process of a certain form. For example, a process providing the type $A \tensor B$ first sends out a channel satisfying $A$, then acts as $B$. Similarly, a process offering $\terminate$ sends the label $\irb{end}$ and terminates. We call these \emph{structural types} since they correspond to processes of a certain structure. In this paper, we extend the base type system with intersections and unions. We call these \emph{property types} since they do not correspond to specific forms of processes in that any process may be assigned such a type. In addition, if we interpret a type as specifying a property, then intersection corresponds to satisfying two properties simultaneously and union corresponds to satisfying one or the other.

Our goal is to show that the base system extended with intersection, unions, recursive types, and a natural notion of subtyping is type-safe. We do this by proving the usual type preservation and progress theorems, which correspond to session fidelity and deadlock freedom in the concurrent context. In the presence of a strong subtyping relation and transparent (i.e.\ non-generative)  equirecursive types, intersections and unions turn out to be powerful enough to specify many interesting communications behaviors, which we demonstrate with examples analogous to those in functional languages \cite{FreemanP91,Dunfield03}.

Our contributions are summarized below:
\begin{itemize}
  \item We introduce intersection and union types to a session-typed concurrent calculus and prove session fidelity and deadlock freedom.
  \item We give a simple and sound coinductive subtyping relation in the presence of equirecursive types, intersections, and unions reminiscent of Gentzen's multiple conclusion sequent calculus \cite{Gentzen35, Girard87}.
  \item We show how intersections and unions can be used as refinements of recursive types in a linear setting.
  \item We show decidability of subtyping and present a system for algorithmic type checking (the proof of its soundness and completeness is in progress at the time of submission).
  \item We demonstrate how internal and external choice can be understood as singletons interacting with intersection and union.
\end{itemize}

