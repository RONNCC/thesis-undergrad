
\chapter{Introduction}

A concurrent system consists of processes that work together to compute a result. In the message-passing formulation of concurrency, interaction between processes is established by exchanging messages through channels that go between them. Session types are assigned to channels in a type safe setting to prescribe the communication behavior along channels. The processes at each end of a channel must respect this type when using the said channel.

Recently, session-typed message-passing concurrency has been put on the firm foundations of logic by establishing a Curry-Howard correspondence to intuitionistic linear sequent calculus \cite{CairesP10, PfenningG15, Honda93}. In this formulation, linear propositions are interpreted as session-types, proofs as processes, and cut elimination as communication. The basic type system described in prior work is enough to guarantee strong properties such as deadlock freedom and session fidelity, and the theory can accommodate many different types of communication such as synchronous or asynchronous (we stick to synchronous communication to keep things simple). On the other hand, many interesting behavioral properties turn out to be inexpressible using the basic type system. This usually leads to having to implement impossible cases (which cannot offer any interesting behavior and generally terminate the program). The type checker cannot verify that these cases are in fact impossible, which is not only annoying but generally leads to errors when those cases turn out to be relevant.

In this thesis, we extend the type system of \cite{PfenningG15} with intersection and union types in order to specify and statically verify more interesting behavioral properties of processes. Previously, Freeman and Pfenning has shown how intersections can be used as refinements in a conventional functional language \cite{FreemanP91}. We hope to carry their success into the concurrent setting and expand on it by introducing unions as well.

To do so, we would like to show that the base system extended with intersections, unions, recursive types, and a natural notion of subtyping is type-safe, and the resulting type system admits a type checking algorithm. We accomplish the latter by giving an algorithmic system and showing its equivalence to the declarative one. The former would follow from the usual type preservation and progress theorems, which correspond to session fidelity and deadlock freedom in the concurrent context. We have not yet been able to prove type-safety, though we conjecture that it holds and are working on a proof.

In the presence of a strong subtyping relation and transparent (i.e.\ non-generative)  equirecursive types, intersections and unions turn out to be powerful enough to specify many interesting communications behaviors, which we demonstrate with examples analogous to those in functional languages \cite{FreemanP91,Dunfield03}.

The rest of the thesis is structured as follows. In \cref{base}, we give a brief overview of prior work on the correspondence between concurrent computation and linear sequent logic. We introduce the base system and equirecursive types. \Cref{base:subtyping} extends this system with a natural notion of subtyping. In \cref{refinements}, we add intersection and union types to the system. This necessitates modifying the subtyping relation in order to admit distributivity laws. We present the new subtyping relation and explore its properties. \Cref{algorithmic} presents the algorithmic system, and finally, \cref{conclusion} concludes with suggestions for future work.

