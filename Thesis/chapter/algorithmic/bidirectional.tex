
\section{Bidirectional Type-checking}
\label{algorithmic:bidirectional}

Designing a syntax directed type checking algorithm is quite simple for the base system where we only have structural types (no recursion or subtyping), since the form of the process determines a unique applicable typing rule. The $\cut$ rule causes a small problem since we do not have a type for the helper process to check against. This is solved by adding type annotations in spawning processes so that the new form is $\tspawnType c {P_c} A {Q_c}$. We define $\erase{P}$ to be the function which erases these annotations.

In the extended system with subtyping and property types, type-checking is trickier for two reasons: (1) subsumption can be applied anytime where one of the types in $A \le B$ is free, and (2) intersection left and union right rules lose information which means they have to be applied non-deterministically. The latter issue is resolved by switching to a multiset context multiple conclusion logic just like we did with subtyping. This makes intersection left and union right rules invertible, so they can be applied eagerly.

The former problem is solved by \emph{bidirectional type-checking} where we only check subtyping at the identity rule and at recursive process variables (delegation). This relies on the subformula property for the sequent calculus, excepting only the cut rule which is annotated. The fact that we can delay subtyping in this way is formalized later in \cref{algorithmic:delegation-sub}, when we prove the equivalence of the algorithmic system to the declarative one.

Algorithmic typing rules for processes are given in \cref{algorithmic:process-typing}. The rules make use of the following definitions:

\begin{definition}[Cumulative Intersection and Union]
  For all non-empty $\typeList = A_1, \ldots, A_n$, define $\all \typeList = A_1 \intersect \ldots \intersect A_n$ and $\any \typeList = A_1 \union \ldots \union A_n$. Even though we consider $\typeList$ to be unordered, these operations are well-defined since $\intersect$ and $\union$ are associative and commutative (with respect to the subtyping relation).

  Similarly, for a context $\ctx = c_1 : \typeList_1, \ldots, c_n : \typeList_n$, define $\all \ctx = c_1 : \all {\typeList_1}, \ldots, c_n : \all {\typeList_n}$ and $\any \ctx = c_1 : \any {\typeList_1}, \ldots, c_n : \any {\typeList_n}$.
\end{definition}

\begin{definition}[Subtyping of Contexts]
  For $\ctx = c_1 : \typeList_1, \ldots, c_n : \typeList_n$ and $\ctx' = c_1 : \typeListB_1, \ldots, c_n : \typeListB_n$, we say $\ctx \subA \ctx'$ whenever $\typeList_i \subA \all \typeListB_i$ for all $1 \le i \le n$.
\end{definition}


The rules $\intersect\Right$, $\intersect\Left$, $\union\Right$, $\union\Left$, $\mu\Right$ and $\mu\Left$ are applied eagerly as mentioned before. If and when these rules are saturated%
\footnote{$\size$ of the context or the provided type decreases by one after each application of these rules. Since we only consider contractive types, $\size$ is finite and we must reach all structural types at some point.}%
, we examine the structure of the process and non-deterministically commit to a type. Whenever we need to split the context, we again try all possible splits.%
\footnote{One might be tempted to think the correct split can be computed based on the free channels in the components of $P$. Unfortunately, that approach does not always work since case statements might have unused branches. These extra branches are never type-checked, so they could contain free occurrences of any channel name.}
At every step, either the process gets structurally smaller, or it stays the same and the $\size$ of a type in the context or the provided type shrinks by 1. Thus, the algorithm always terminates.

\begin{rules}[algorithmic:process-typing]{Algorithmic process typing}
  % Intersection
  \infer[\intersect\Right]{\typeRecAJR \ctx P c {A \intersect B, \typeList}}
    { \typeRecAJR \ctx P c {A, \typeList}
    & \typeRecAJR \ctx P c {B, \typeList}
    }
  \and \infer[\intersect\Left]{\typeRecAJR{\ctx, c : (\typeList, A \intersect B)}{P}{d}{\typeListB}}
    {\typeRecAJR{\ctx, c : (\typeList, A, B)}{P}{d}{\typeListB}}
  % Union
  \\ \infer[\union\Right]{\typeRecAJR \ctx P c {A \union B, \typeList}}
    {\typeRecAJR \ctx P c {A, B, \typeList}}
  \and \infer[\union\Left]{\typeRecAJR{\ctx, c : (\typeList, A \union B)} P d \typeListB}
    { \typeRecAJR {\ctx, c : (\typeList, A)} P d \typeListB
    & \typeRecAJR {\ctx, c : (\typeList, B)} P d \typeListB
    }
  % Recursive
  \\ \infer[\mu\Right]{\typeRecAJ P c {\recursive t A, \typeList}}
    { \typeRecAJ P c {\subst {\recursive t A} t A, \typeList} }
  \and \infer[\mu\Left]{\typeRecAJR {\ctx, c : (\typeList, \recursive t A)} P d {\typeListB}}
    { \typeRecAJR {\ctx, c : (\typeList, \subst {\recursive t A} t A)} P d {\typeListB} }
  % id and cut
  \\ \infer[\id]{ \typeRecAJR {c : \typeList} {\tfwd d c} {d} {\typeListB} }
    { \typeList \subA \typeListB }
  \and \infer[\cut]{ \typeRecAJR {\ctx, \ctx'} {\tspawnType c {P_c} A {Q_c}} {d} \typeList }
    { \typeRecAJR \ctx {P_c} {c} {A}
    & \typeRecAJR {\ctx', c : A} {Q_c} {d} {\typeList}
    }
  % Terminate
  \\ \infer[\terminate\Right]{\typeRecAJR{\emptyCtx}{\tclose c}{c}{\terminate, \typeList}}
   {}
  \and \infer[\terminate\Left]{\typeRecAJR{\ctx, c : (\typeList, \terminate)}{\twait c P} d \typeListB}
    { \typeRecAJR {\ctx} P d \typeListB}
  % Tensor
  \and \infer[\tensor\Right]{\typeRecAJR{\ctx, \ctx'}{\tsend c d {P_d} Q }{c}{A \tensor B, \typeList}}
    { \typeRecAJR \ctx P d A
    & \typeRecAJR {\ctx'} Q c B
    }
  \and \infer[\tensor\Left]{ \typeRecAJR{\ctx, c : (\typeList, A \tensor B)}{\trecv{d}{c}{P_d}}{e}{\typeListB} }
    { \typeRecAJR{\ctx, d : A, c : B}{P_d}{e}{\typeListB} }
  % Internal choice
  \and \infer[\internal\Right]{\typeRecAJR \ctx { \tselect{c}{i}{P} } {c} {\internals{A}{I}, \typeList }}
    { i \in I
    & \typeRecAJR \ctx {P}{c}{A_i}
    }
  \and \infer[\internal\Left]{ \typeRecAJR { \ctx, c : (\typeList, \internals{A}{I}) } { \tcase{c}{\tbranches{P}{J}} } {d} {\typeListB} }
   { I \subseteq J
   & \typeRecAJR{\ctx, c : A_k}{P_k}{d}{\typeListB}~\text{for}~k\in I
   }
  % Lolli
  \and \infer[\lolli\Right]{ \typeRecAJR{\ctx}{\trecv{d}{c}{P_d}}{c}{A \lolli B, \typeList} }
    { \typeRecAJR{\ctx, d : A}{P_d}{c}{B} }
  \and \infer[\lolli\Left]{\typeRecAJR{\ctx, \ctx', c : (\typeList, A \lolli B)}{ \tsend{c}{d}{P_d}{Q} } {e}{\typeListB}}
    { \typeRecAJR{\ctx}{P_d}{d}{A}
    & \typeRecAJR{\ctx', c : B}{Q}{e}{\typeListB}
    }
  % External choice
  \and \infer[\external\Right]{ \typeRecAJR \ctx { \tcase{c}{\tbranches{P}{I}} } {c} {\externals{A}{J}, \typeList} }
   { J \subseteq I
   & \typeRecAJR \ctx {P_k}{c}{A_k}~\text{for}~k\in J
   }
  \and \infer[\external\Left]{\typeRecAJR{\ctx, c : (\typeList, \externals{A}{I})} { \tselect{c}{i}{P} } {d} {\typeListB}}
    { i \in I
    & \typeRecAJR{\ctx, c : A_i}{P}{d}{\typeListB}
    }
  % Recursive processes
  \\ \infer[\rec]{\typeRecAJ {\tapp {\parens*{ \trec p {\tvector y} P}} {\tvector z}} c \typeList}
   { \typeRecD \ctx {\recCtx'} {\subst {\tvector z} {\tvector y} P} c \typeList
   & \recCtx' = \recCtx, \typeD {\subst {\tvector y} {\tvector z} \ctx} {\tvar p {\tvector y}} {\subst {\tvector y} {\tvector z} c} \typeList
   }
   \and \infer[\procVar]{\typeRecA {\ctx'} \recCtx {\tapp p {\tvector z}} {\subst {\tvector z} {\tvector y} c} \typeListB}
    {\parens{\typeDJ {\tvar p {\tvector y}} c \typeList} \in \recCtx
    & \ctx' \subA \subst {\tvector z} {\tvector y} \ctx
    & \any\typeList \subA \typeListB
    }
\end{rules}

