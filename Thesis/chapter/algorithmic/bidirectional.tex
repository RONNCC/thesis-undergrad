
\section{Bidirectional Type-checking}

Designing a syntax directed type checking algorithm is quite simple for the base system where we only have structural types (no recursion or subtyping), since the form of the process determines a unique applicable typing rule. The $\cut$ rule causes a small problem since we do not have a type for the helper process to check against. This is solved by adding type annotations in spawning processes so that the new form is $\tspawnType c {P_c} A {Q_c}$. We define $\erase{P}$ to be the function which erases these annotations.

In the extended system with subtyping and property types, type-checking is trickier for two reasons: (1) subsumption can be applied anytime where one of the types in $A \le B$ is free, and (2) intersection left and union right rules lose information which means they have to be applied non-deterministically. The latter issue is resolved by switching to a multiset context multiple conclusion logic just like we did with subtyping. This makes intersection left and union right rules invertible, so they can be applied eagerly.

The former problem is solved by \emph{bidirectional type-checking} where we only check subtyping at the identity rule (delegation). This relies on the subformula property for the sequent calculus, excepting only the cut rule which is annotated. \todo{Explain the subformula rule.}
%This relies on the following crucial observation: if we are checking a process against a structural type, subparts of the process must check against subparts of the type. For example, if we are checking $\tsend c d {P_d} Q$ against $c : A \tensor B$, then $P_d$ must check against $d : A$ and $Q$ must check against $c : B$. To make sure we can always decompose property types, we add unfolding rules on the right and on the left, and apply these eagerly along with rules for intersections and unions.

Algorithmic typing rules for processes are given in \cref{algorithmic:process-typing}. Rules for recursive processes are not duplicated since they simply capture sets of types rather than a single type. \todo{Do rules for recursive processes since the rule for variable needs to be relaxed.}

\begin{rules}[algorithmic:process-typing]{Algorithmic process typing}
  % Intersection
  \infer[\intersect\Right]{\typeRecAJR \ctx P c {A \intersect B, \typeList}}
    { \typeRecAJR \ctx P c {A, \typeList}
    & \typeRecAJR \ctx P c {B, \typeList}
    }
  \and \infer[\intersect\Left]{\typeRecAJR{\ctx, c : (\typeList, A \intersect B)}{P}{d}{\typeListB}}
    {\typeRecAJR{\ctx, c : (\typeList, A, B)}{P}{d}{\typeListB}}
  % Union
  \\ \infer[\union\Right]{\typeRecAJR \ctx P c {A \union B, \typeList}}
    {\typeRecAJR \ctx P c {A, B, \typeList}}
  \and \infer[\union\Left]{\typeRecAJR{\ctx, c : (\typeList, A \union B)} P d \typeListB}
    { \typeRecAJR {\ctx, c : (\typeList, A)} P d \typeListB
    & \typeRecAJR {\ctx, c : (\typeList, B)} P d \typeListB
    }
  % Recursive
  \\ \infer[\mu\Right]{\typeRecAJ P c {\recursive t A, \typeList}}
    { \typeRecAJ P c {\subst {\recursive t A} t A, \typeList} }
  \and \infer[\mu\Left]{\typeRecAJR {\ctx, c : (\typeList, \recursive t A)} P d {\typeListB}}
    { \typeRecAJR {\ctx, c : (\typeList, \subst {\recursive t A} t A)} P d {\typeListB} }
  % id and cut
  \\ \infer[\id]{ \typeRecAJR {c : \typeList} {\tfwd d c} {d} {\typeListB} }
    { \typeList \subA \typeListB }
  \and \infer[\cut]{ \typeRecAJR {\ctx, \ctx'} {\tspawnType c {P_c} A {Q_c}} {d} \typeList }
    { \typeRecAJR \ctx {P_c} {c} {A}
    & \typeRecAJR {\ctx', c : A} {Q_c} {d} {\typeList}
    }
  % Terminate
  \\ \infer[\terminate\Right]{\typeRecAJR{\emptyCtx}{\tclose c}{c}{\terminate, \typeList}}
   {}
  \and \infer[\terminate\Left]{\typeRecAJR{\ctx, c : (\typeList, \terminate)}{\twait c P} d \typeListB}
    { \typeRecAJR {\ctx} P d \typeListB}
  % Tensor
  \and \infer[\tensor\Right]{\typeRecAJR{\ctx, \ctx'}{\tsend c d {P_d} Q }{c}{A \tensor B, \typeList}}
    { \typeRecAJR \ctx P d A
    & \typeRecAJR {\ctx'} Q c B
    }
  \and \infer[\tensor\Left]{ \typeRecAJR{\ctx, c : (\typeList, A \tensor B)}{\trecv{d}{c}{P_d}}{e}{\typeListB} }
    { \typeRecAJR{\ctx, d : A, c : B}{P_d}{e}{\typeListB} }
  % Internal choice
  \and \infer[\internal\Right]{\typeRecAJR \ctx { \tselect{c}{i}{P} } {c} {\internals{A}{I}, \typeList }}
    { i \in I
    & \typeRecAJR \ctx {P}{c}{A_i}
    }
  \and \infer[\internal\Left]{ \typeRecAJR { \ctx, c : (\typeList, \internals{A}{I}) } { \tcase{c}{\tbranches{P}{J}} } {d} {\typeListB} }
   { I \subseteq J
   & \typeRecAJR{\ctx, c : A_k}{P_k}{d}{\typeListB}~\text{for}~k\in I
   }
  % Lolli
  \and \infer[\lolli\Right]{ \typeRecAJR{\ctx}{\trecv{d}{c}{P_d}}{c}{A \lolli B, \typeList} }
    { \typeRecAJR{\ctx, d : A}{P_d}{c}{B} }
  \and \infer[\lolli\Left]{\typeRecAJR{\ctx, \ctx', c : (\typeList, A \lolli B)}{ \tsend{c}{d}{P_d}{Q} } {e}{\typeListB}}
    { \typeRecAJR{\ctx}{P_d}{d}{A}
    & \typeRecAJR{\ctx', c : B}{Q}{e}{\typeListB}
    }
  % External choice
  \and \infer[\external\Right]{ \typeRecAJR \ctx { \tcase{c}{\tbranches{P}{I}} } {c} {\externals{A}{J}, \typeList} }
   { J \subseteq I
   & \typeRecAJR \ctx {P_k}{c}{A_k}~\text{for}~k\in J
   }
  \and \infer[\external\Left]{\typeRecAJR{\ctx, c : (\typeList, \externals{A}{I})} { \tselect{c}{i}{P} } {d} {\typeListB}}
    { i \in I
    & \typeRecAJR{\ctx, c : A_i}{P}{d}{\typeListB}
    }
\end{rules}

