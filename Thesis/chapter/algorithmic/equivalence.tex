
\section{Equivalence to the Declarative System}
\label{algorithmic:equivalence}

We are now ready to prove that the algorithmic system is equivalent (modulo type annotations) to the declarative system. We do this by showing the soundness and completeness of the algorithmic system with respect to the declarative system.

% Note that subtyping extends to contexts in a straightforward way: $\ctx \subASing \ctx'$ if and only if $\ctx = c_1 : \typeList_1, \ldots, c_n : \typeList_n$, $\ctx' = c_1 : \typeListB_1, \ldots, c_n : \typeListB_n$, and $\typeList_i \subA \all{\typeListB_i}$ for all $i \in \set{1, \ldots, n}$.

\subsection{Soundness}

\begin{theorem}[Soundness of Algorithmic Typing]
  If $\typeRecAJ P c \typeList$, then $\typeRecD {\all\ctx} {\recCtx'} {\erase P} c {\any \typeList}$ where $\recCtx'$ is $\recCtx$ suitably converted using $\all$ and $\any$.
\end{theorem}
\begin{proof}
  We proceed by induction on the typing derivation.
  \begin{description}
    % Intersections
    \item[Case $\intersect\Right :$] This is one of the two tricky cases (the other is the dual $\union\Left$). In fact, this case is the reason we needed distributivity of intersection over union. Assume $\DD : \typeRecAJR \ctx P c {A, \typeList}$ and $\EE : \typeRecAJR \ctx P c {B, \typeList}$. We have the following derivation:
    $$ \infer[\irb{Sub}\Right]{\typeRecAJR {\all\ctx} P c {\any{\parens{A \intersect B, \typeList}}}}
        { \infer[\intersect\Right]{\typeRecAJR {\all\ctx} P c {(A \union \any\typeList) \intersect (B \union \any\typeList)}}
           { \deduce[\induction{(\DD)}]{\typeRecAJR {\all\ctx} P c {A \union \any\typeList}} {}
           & \deduce[\induction{(\EE)}]{\typeRecAJR {\all\ctx} P c {B \union \any\typeList}} {}
           }
        & \deduce[\text{\Cref{refinements:distributivity-proof}}]{\ldots \subASing \ldots}{}
        }
    $$
    where \cref{refinements:distributivity-proof} is used to prove $(A \union \any\typeList) \intersect (B \union \any\typeList) \subASing (A \intersect B) \union \any\typeList$.

    \item[Case $\intersect\Left :$] Follows from $\irb{Sub}\Left$ (we need subtyping to reorder the large intersection) and the induction hypothesis.

    % Unions
    \item[Case $\union\Right :$] Follows from $\irb{Sub}\Right$ (to reorder the large union) and the induction hypothesis.
    \item[Case $\union\Left :$] Similar to $\intersect\Right$.

    % Recursive types
    \item[Case $\mu\Right, \mu\Left :$] Follows immediately from $\irb{Sub}\Right$ and $\irb{Sub}\Left$, respectively.

    % id and cut
    \item[Case $\id :$] Follows by an application of $\irb{Sub}\Right$ and $\id$. We note that if $\typeList \subA \typeListB$, then $\all\typeList \subASing \any\typeListB$ by a trivial induction on the sum of the number of types in $\typeList$ and $\typeListB$.
    \item[Case $\cut :$] Follows immediately from $\cut$ and the two induction hypotheses.

    % Structural rules
    \item[Case $\terminate\Right, \tensor\Right, \internal\Right, \lolli\Right, \external\Right :$] Follows from $\irb{Sub}\Right$ to ``pick out the right type'' followed by the corresponding rule in the declarative system.
    \item[Case $\terminate\Left, \tensor\Left, \internal\Left, \lolli\Left, \external\Left :$] Follows from $\irb{Sub}\Left$ to ``pick out the right type'' followed by the corresponding rule in the declarative system.
    \item[Case $\mu :$] Follows immediately from the induction hypothesis.
    \item[Case $\procVar :$] Follows from $\irb{Sub}\Right$, $\irb{Sub}\Left$, and the induction hypothesis.
  \end{description}
\end{proof}


\subsection{Completeness}

\begin{lemma}[Completeness of Delayed Subtyping]
  \label{algorithmic:delegation-sub}
  \todo{What to call this?}
  The following are admissible:
  \begin{itemize}
    \item If $\typeRecAJ P c \typeList$ and $\typeList \subA \typeList'$ then $\typeRecAJ P c {\typeList'}$.
    \item If $\typeRecAJR {\ctx, d : \typeList} P c \typeListB$ and $\typeList' \subA \typeList$ then $\typeRecAJR {\ctx, d : \typeList'} P c \typeListB$.
  \end{itemize}
  Note that $P$ stays the same, which means type annotations do not need to change.
\end{lemma}
\todo{Do the lemma.}
%   By induction on the structure of $P$ and appealing to \cref{algorithmic:interpretation-commutes}. Since both parts of the theorem are similar, we only show the former. Assume $\typeRecAJ P c \typeList$ and $\typeList \subA \typeList'$. By \cref{algorithmic:interpretation-commutes}, we know
%   $$ \interpP {\typeRecAJ P c \typeList}
%     \iff \interp \typeList {\lam A { \interpC \ctx {\lam {\ctx'} {\typeRecAJRStruct {\ctx'} P c A}} }}
%   $$
%   and
%   $$ \interpP {\typeList \subA \typeList'}
%      \iff \interpC \typeList {\lam A {\interp {\typeList'} {\lam {A'} {A \subAStruct A'}} }}.
%   $$
%   By \cref{algorithmic:interpretation-mix}, it suffices to show
%   $$
%     \typeRecAJStruct P c A \wedge \interp {\typeList'} {\lam {A'} {A \subAStruct A'}}
%     \implies \interp {\typeList'} {\lam {A'} {\typeRecAJStruct P c {A'}} }
%   $$
%   for some $A \in \typesStruct$. Similarly, by \cref{algorithmic:interpretation-connectives}, it suffices to show
%   $$
%     \typeRecAJStruct P c A \wedge A \subAStruct A'
%     \implies \typeRecAJStruct P c {A'}
%   $$
%   for some $A' \in \typesStruct$. Assume $\DD : \typeRecAJStruct P c A$ and $\EE : A \subAStruct A'$. We use inversion on $\DD$:
%   \begin{description}
%     \item[Case $\id :$] Follows from the transitivity of $\subA$.
%     \item[Case $\cut :$] We have
%       $$ \DD = \vcenter{
%           \infer[\cut]{\typeRecAJR {\ctx_1, \ctx_2} {\tspawnType d Q B {P'}} c A}
%            { \DD_1 : \typeRecAJR {\ctx_1} Q d B
%            & \DD_2 : \typeRecAJR {\ctx_2, d : B} Q d A
%            }
%          }
%       $$
%       The induction hypothesis on $\DD_2$ and $\EE$ gives $\typeRecAJR {\ctx_2, d : B} Q d {A'}$. $\cut$ on this and $\DD_1$ gives the result.
%     \item[Case $\terminate\Right, \tensor\Right, \internal\Right, \lolli\Right, \external\Right :$] Inversion on $\EE$ gives the necessary subtyping relation(s). We match these with the sub-derivations from $\DD$ which lets us apply the induction hypotheses. The result follows from the same rule using these.
%     \item[Case $\terminate\Left, \tensor\Left, \internal\Left, \lolli\Left, \external\Left :$] Follows immediately from the induction hypotheses.

%     \item[Case $\mu, \procVar :$] \todo{Recursive processes.}
%   \end{description}
% \end{proof}


\begin{theorem}[Completeness of Algorithmic Typing]
  If $\typeRecDJ P c A$, then there exists $P'$ such that $\erase{P'} = P$ and $\typeRecAJ {P'} c A$.
\end{theorem}
\begin{proof}
  We proceed by induction on the typing derivation.
  \begin{description}
    % id and cut
    \item[Case $\id :$] We use $\id$ and the fact that $\subASing$ is reflexive.
    \item[Case $\cut :$] Follows from $\cut$ and the two induction hypotheses. We use the type from the derivation of the first branch to annotate $P$.

    % Structural rules
    \item[Case $\terminate\Right, \tensor\Right, \internal\Right, \lolli\Right, \external\Right :$] Follows immediately from the corresponding rule and the induction hypotheses.
    \item[Case $\terminate\Left, \tensor\Left, \internal\Left, \lolli\Left, \external\Left :$] Follows immediately from the corresponding rule and the induction hypotheses.

    % Subtyping
    \item[Case $\irb{Sub}\Right, \irb{Sub}\Left :$] Follows from the induction hypothesis and \cref{algorithmic:delegation-sub}.

    % Intersection and union
    \item[Case $\intersect\Right :$] The induction hypotheses give the two premises needed to apply the corresponding rule in the algorithmic system. The only problem is, the annotated processes returned by the two hypotheses need not have the same annotations. We believe taking the intersection of corresponding annotations should be sufficient, though we could not come up with a proof. \todo{What to do for this case?}
    \item[Case $\union\Left :$] Similar to the previous case. \todo{Related to the previous case.}

    % Recursive processes
    \item[Case $\mu, \procVar :$] Follows immediately from the induction hypotheses.
  \end{description}
\end{proof}

