
\section{Type Safety}
\label{algorithmic:semantics}

In this \namecref{algorithmic:semantics}, we prove type safety for the algorithmic system. Through the equivalence of the algorithmic system to the declarative system, we also establish type safety for the declarative system.

Note that we did not add new forms of processes, so process configurations and reduction are the same as in \cref{base:semantics}. We have, however, altered process typing quite a bit. This will require a change in the structure of progress and preservation proofs. We also slightly modify configuration typing to use the new rules:
  $$
  \infer[\confOne]{\provides{\config, \proc c P}{c}{A}}
   { \typeRecA \ctx \emptyset P c A
   & \providesCtx \config \ctx
   }
  $$
Note that we use singular types (rather than multisets of types) at the top level. The other cases ($\confZero$ and $\confN$) are the same as before. Even though we use the algorithmic typing judgement in this definition, the results below will hold for declarative typing since these systems are equivalent.


\subsection{Progress}

We need to use a slightly different inversion \namecref{algorithmic:process-inversion} than we did in \cref{base:progress-section}:
\begin{lemma}[Process Inversion]
  \label{algorithmic:process-inversion}
  If $\typeRecA {\ctx} {\recCtx} P c A$, $\typeRecA {\ctx', c : A} {\recCtx} Q d \typeListB$, and $\poised {\proc c P}$, then the following hold:
  \begin{itemize}
    \item If $Q = \twait d {Q'}$ then $P = \tclose c$.
    \item If $Q = \trecv x d {Q'}$ then $P = \tsend c e {R_e} {P'}$.
    \item If $Q = \tsend d e {R_e} {Q'}$ then $P = \trecv x c {P'}$.
    \item If $Q = \tcase d {\tbranches {Q'} I}$ then $P = \tselect c i {P'}$ and $i \in I$.
    \item If $Q = \tselect d i {Q'}$ then $P = \tcase c {\tbranches {P'} I}$ and $i \in I$.
  \end{itemize}
\end{lemma}
\begin{proof}
  Let $\DD : \typeRecA {\ctx} {\recCtx} P c A$ and $\EE : \typeRecA {\ctx', c : A} {\recCtx} Q d \typeListB$. Since we are only interested in cases where $\proc d Q$ is communicating along $c$, we can assume $\blockedOn {\proc d Q} c$.%
\footnote{Otherwise, inversion on $Q$ shows that all of the equalities in the statement of the \namecref{algorithmic:process-inversion} are false. Thus, the result follows vacuously.} We proceed by induction on $\size\ctx + \size A + \size{\ctx'} + \size\typeListB$.
  \begin{itemize}
    \item If $\ctx$, $\ctx'$, or $\typeListB$ contains a non-structural type, we use \cref{algorithmic:property-invertible} to break it down and apply the induction hypothesis on the derivation we get. In cases where there are two derivations, we pick either.
    \item If $A$ is a $\mu$, we use \cref{algorithmic:property-invertible} on $\DD$ and on $\EE$, and apply the induction hypothesis.
    \item Otherwise, we have $\structural\ctx$, $\structural{\ctx'}$, and $\structural\typeListB$, and we know $A$ is not a $\mu$.
    \begin{itemize}
      \item If $A = A_1 \intersect A_2$, \cref{algorithmic:property-invertible-extended-corollary} on $\EE$ gives either $\EE_1 : \typeRecA {\ctx', c : A_1} {\recCtx} Q d \typeListB$ or $\EE : \typeRecA {\ctx', c : A_2} {\recCtx} Q d \typeListB$. In the former case, inversion on $\DD$ gives $\DD_1 : \typeRecA {\ctx} {\recCtx} P c {A_1}$ and we can immediately apply the induction hypothesis on $\DD_1$ and $\EE_1$. The latter case is symmetric.
      \item If $A = A_1 \union A_2$, then we apply \cref{algorithmic:property-invertible-extended-corollary} on $\DD$ this time and use inversion on $\EE$. We get matching cases and apply the induction hypothesis.
      \item Otherwise, $A$ must be a structural type, $\DD$ must be by a structural rule on the right, and $\EE$ must be by a structural rule on $c$. Inversion on $\EE$ and the form of $Q$ forces the form of $A$ (e.g.\ if $Q = \twait d {Q'}$ then $A = 1$). Then, inversion on $\DD$ reveals the form of $P$, and the form of $A$ forces it to be complementary to the form of $Q$.
    \end{itemize}
  \end{itemize}
\end{proof}


Given \cref{algorithmic:process-inversion}, progress follows quite trivially:

\begin{theorem}[Progress]
If $\providesCtx \config \ctx$ then either
\begin{enumerate}
  \item $\steps{\config}{\config'}$ for some $\config'$, or
  \item $\config$ is poised.
\end{enumerate}
\end{theorem}
\begin{proof}
  We proceed by induction on the derivation of $\providesCtx \config \ctx$. The case for multiple channels follows immediately from the induction hypotheses. For the single channel case, we know $\config = \config', \proc c P$. By inversion, $\DD : \typeRecA {\ctx_c} \emptyset P c A$ and $\EE : \providesCtx {\config'} {\ctx_c}$. By the induction hypothesis, either $\config'$ takes a step, in which case $\config$ takes a step and we are done, or $\config'$ is poised. Assume $\config'$ is poised. We case on the structure of $P$:
  \begin{itemize}
    \item If $P$ is a forward, a spawn, or a recursive process then $\config$ steps by $\irb{id}$, $\irb{cut}$, or $\irb{rec}$, respectively.
    \item If $\poised {\proc c P}$, then $\config$ is poised.
    \item Otherwise, we must have $\blockedOn {\proc c P} d$ where $c \neq d$. By \cref{algorithmic:channel-in-context}, $d : B \in \ctx_c$. Inversion on $\EE$ gives $\typeRecAJR {\ctx_d} Q d B$ where $\proc d Q \in \config'$. Since $\config'$ is poised, $\proc d Q$ is poised. We can then apply \cref{algorithmic:process-inversion}, which shows that $P$ and $Q$ have matching forms and can take a step together.
  \end{itemize}
\end{proof}


\section{Type Preservation}

First, we need the following inversion results.

\begin{lemma}[Inversion of Id]
  \label{algorithmic:id-inversion}
If $\typeRecAJ {\tfwd c d} c \typeList$, then $\ctx = d : \typeListB$ and $\typeListB \subA \typeList$.
\end{lemma}
\begin{proof}
  By induction on the typing derivation. The case for $\id$ is immediate. Otherwise, the last rule must be one of $\intersect\Right$, $\intersect\Left$, $\union\Right$, $\union\Left$, $\mu\Right$, or $\mu\Left$, in which case we immediately apply the induction hypotheses and combine the results with the corresponding subtyping rule.
\end{proof}

\begin{lemma}[Inversion of Cut]
  \label{algorithmic:cut-inversion}
  If $\typeRecAJ {\tspawnType d {P_d} A {Q_d}} c \typeList$ then there exist $\ctx_1$ and $\ctx_2$ such that $\ctx = \parens{\ctx_1, \ctx_2}$, $\typeRecAJR {\ctx_1} {Q_d} d A$, and $\typeRecAJR {\ctx_2, d : A} {P_d} c \typeList$.
\end{lemma}
\begin{proof}
  Let $P = \tspawnType d {Q_d} A {P'_d}$ and $\DD : \typeRecAJ P c \typeList$. We proceed by induction on $\DD$.
  \begin{itemize}
    \item If $\DD$ is by $\cut$, we have the desired result.
    \item If $\DD$ is by $\intersect\Left$ or $\mu\Left$ on channel $e \in \dom\ctx$, we apply the induction hypothesis to get two derivations. The results follows from the same rule on one of the derivations (depending on whether $e \in \dom{\ctx_1}$ or $e \in \dom{\ctx_2}$).
    \item If $\DD$ is by $\union\Right$ or $\mu\Right$, the result follows from the induction hypothesis and the same rule on the typing derivation for $P'_d$.
    \item Otherwise, $\DD$ is by $\intersect\Right$ or $\union\Left$. Since the case for $\union\Left$ is slightly more complicated, we will show that. We have:
    $$ \DD =
       \vcenter{
         \infer[\union\Left]{\typeRecAJR{\ctx', e : (\typeListB, B \union C)} P c \typeList}
           { \DD_1 : \typeRecAJR {\ctx', e : (\typeListB, B)} P c \typeList
           & \DD_2 : \typeRecAJR {\ctx', e : (\typeListB, C)} P c \typeList
           }
       }
    $$
    Induction hypothesis on $\DD_1$ gives $\ctx^B_1, \ctx^B_2$ such that $\ctx', e : (\typeListB, B) = \parens{\ctx^B_1, \ctx^B_2}$, $\EE_1 : \typeRecAJR {\ctx^B_1} {Q_d} d A$ and $\EE_2 : \typeRecAJR {\ctx^B_2, d : A} {P'_d} c \typeList$.
    Similarly, from $\DD_2$ we get $\ctx^C_1, \ctx^C_2$ such that $\ctx', e : (\typeListB, C) = \parens{\ctx^C_1, \ctx^C_2}$, $\FF_1 : \typeRecAJR {\ctx^C_1} {Q_d} d A$ and $\FF_2 : \typeRecAJR {\ctx^C_2, d : A} {P'_d} c \typeList$. By \cref{refinements:regularity-corollary}, we know $\dom {\ctx_1^B} = \dom{\ctx_1^C}$ and $\dom {\ctx_2^B} = \dom{\ctx_2^C}$, so we get the desired results by an application of $\union\Left$ on $\EE_1$ and $\FF_1$ or $\EE_2$ and $\FF_2$, depending on whether $e \in \ctx_1^B$ or $e \in \ctx_2^B$.
 \end{itemize}
\end{proof}


Proof of preservation is a lot more cumbersome, as we have to roll a different induction for each case.
\begin{theorem}[Preservation]
If $\providesCtx \config \ctx$ and $\steps{\config}{\config'}$ then $\providesCtx {\config'} \ctx$.
\end{theorem}
\begin{proof}
  By \cref{framing}, it suffices to consider the subtree which types the root of reduction. So, assume $\steps{\config_1}{\config_2}$ and $\provides {\config_1} c A$ where $c$ is the root of $\steps{\config_1}{\config_2}$. We need to show $\provides {\config_2} c A$.

  By inversion, $\config_1 = \parens{\config_c, \proc c P}$, $\DD : \typeRecA {\ctx_c} \emptyset P c A$, and $\EE :\;\providesCtx {\config_c} {\ctx_c}$. We proceed by case analysis on $\steps{\config_1}{\config_2}$.
  \begin{description}
    \item[\irb{id} :] $P = \tfwd c d$ and $\config_2 = \subst c d {\config_c}$. By \cref{algorithmic:id-inversion}, $\ctx_c = d : D$ where $D \subASing A$. From \cref{algorithmic:delegation-sub} and $\EE$, we know $\provides {\config_c} d A$, thus, $\provides {\subst c d {\config_c}} c A$.

    \item[\irb{cut} :] $P = \tspawnType x {Q_x} A {P'_x}$ and $\config_2 = \config_c, \proc a {Q_a}, \proc c {P'_a}$ where $a$ is fresh. The result follows straightforwardly from \cref{algorithmic:cut-inversion} and substitution.

    \item[\irb{rec} :] Follows by a standard substitution lemma for process variables.
  \end{description}

  In all other cases, we have $\blockedOn {\proc c P} d$. By \cref{algorithmic:channel-in-context}, $\ctx_c = \ctx_c', d : D$ for some $D$, and $\proc d Q \in \config_c$ for some $Q$.   Inversion on $\EE$ gives $\config_c = \config_c', \config_d, \proc d Q$, $\FF_1 : \typeRecA {\ctx_d} \emptyset Q d D$ and $\FF_2 :\;\providesCtx {\config_d} {\ctx_d}$. We only give a representative subset of all cases:
  \begin{description}
    \item[\irb{one} :] $P = \twait d {P'}$ and $Q = \tclose d$. First, we claim $\ctx_d = \emptyCtx$.
    \begin{proof}
      By induction on $\FF_1$. Cases for property rules follow immediately from the induction hypothesis (by picking either branch for $\intersect\Right$ and $\union\Left$). The only other valid case is $\terminate\Right$, which requires the context to be empty, as desired.
    \end{proof}

    This shows $\config_d = \emptyset$. Next, we show $\typeRecA {\ctx_c'} \emptyset {P'} c A$.
    \begin{proof}
      We would like to use induction on $\DD$, however, we need to generalize the induction hypothesis. So instead, we show: for all $\ctx$, $\typeListB$, and $\typeList$, if $\DD' : \typeRecAJR {\ctx, d : \typeListB} {\twait d {P'}} c \typeList$, then $\typeRecAJ {P'} c \typeList$. Using this with $\DD$ gives the desired result.

      If $\DD'$ is by a property rule on $d$, we immediately apply the induction hypothesis (either one for $\union\Left$). If it is by a property rule on $\ctx$ or $A$, the result follows from the same rule on the induction hypotheses. Otherwise, $\DD'$ must be by $\terminate\Left$, which gives the result immediately.
    \end{proof}

    At this point, we know $\config_2 = \config_c', \proc c {P'}$, which can easily be typed using the previous result.

    \item[\irb{tensor} :] $P = \trecv x d {P_x'}$ and $Q = \tsend d x {R_x} {Q'}$. In addition, $\config_2 = \config_c, \config_d, \proc a {R_a}, \proc d {Q'}, \proc c {P_a'}$ where $a$ is fresh. We claim there exist $\ctx_d^1$, $\ctx_d^2$, $D_1$, and $D_2$ such that $\ctx_d = \parens{\ctx_d^1, \ctx_d^2}$, $\typeRecA {\ctx_d^1} \emptyset {R_a} a {D_1}$, $\typeRecA {\ctx_d^2} \emptyset {Q'} d {D_2}$, and $\typeRecA {\ctx_c', a : D_1, d : D_2} \emptyset {P_a'} c A$.
  \begin{proof}
    By induction on $\size {\ctx_c'} + \size A + \size {\ctx_d} + \size D$ and using $\DD$ and $\FF_1$. We assume $\ctx_c'$, $A$, $D$, and $\ctx_d$ are sufficiently generalized (all of them are made free, and all but $D$ is made into multisets). We use $\typeList$ instead of $A$ in the remainder of the proof to emphasize it is a multiset of types.
    \begin{itemize}
      \item If $\ctx_c'$ or $\typeList$ contains a recursive type, or $\ctx_c'$ contains an intersection, or $\typeList$ contains a union, we use \cref{algorithmic:property-invertible} on $\DD$ and apply the induction hypothesis. The result follows from $\mu\Left$, $\mu\Right$, $\intersect\Left$, or $\union\Right$, respectively.
      \item If $\typeList$ contains an intersection, we use \cref{algorithmic:property-invertible} on $\DD$ and apply the induction hypotheses, which give:
      \begin{mathpar}
        \GG_1 : \typeRecA {\ctx_d^1} \emptyset {R_a} a {D_1^1}
        \and \HH_1 : \typeRecA {\ctx_d^2} \emptyset {Q'} d {D_2^1}
        \and \DD_1 : \typeRecA {\ctx_c', a : D_1^1, d : D_2^1} \emptyset {P_a'} c {A_1, \typeList'}
        \\ \GG_2 : \typeRecA {\ctx_d^3} \emptyset {R_a} a {D_1^2}
        \and \HH_2 : \typeRecA {\ctx_d^4} \emptyset {Q'} d {D_2^2}
        \and \DD_2 : \typeRecA {\ctx_c', a : D_1^2, d : D_2^2} \emptyset {P_a'} c {A_2, \typeList'}
      \end{mathpar}
      By \cref{refinements:regularity-corollary}, $\ctx_d^1 = \ctx_d^3$ and $\ctx_d^2 = \ctx_d^4$. From \cref{algorithmic:typing-weakening} and $\intersect\Left$ on $\DD_1$ and $\DD_2$, we have:
      \begin{mathpar}
        \DD_1' : \typeRecA {\ctx_c', a : D_1^1 \intersect D_1^2, d : D_2^1 \intersect D_2^2} \emptyset {P_a'} c {A_1, \typeList'}
        \\ \DD_2' : \typeRecA {\ctx_c', a : D_1^1 \intersect D_1^2, d : D_2^1 \intersect D_2^2} \emptyset {P_a'} c {A_2, \typeList'}
      \end{mathpar}
      Thus, by an application of $\intersect\Right$ on $\DD_1'$ and $\DD_2'$, we get
      $$\typeRecA {\ctx_c', a : D_1^1 \intersect D_1^2, d : D_2^1 \intersect D_2^2} \emptyset {P_a'} c \typeList.$$
      Moreover, applying $\intersect\Right$ on $\GG_1$ and $\GG_2$, and on $\HH_1$ and $\HH_2$ gives
      \begin{mathpar}
        \typeRecA {\ctx_d^1} \emptyset {R_a} a {D_1^1 \intersect D_1^2}
        \and \typeRecA {\ctx_d^2} \emptyset {Q'} d {D_2^1 \intersect D_2^2}
      \end{mathpar}
      respectively. Picking $D_1 = D_1^1 \intersect D_1^2$ and $D_2 = D_2^1 \intersect D_2^2$ gives the result.
      \item The case where $\ctx_c'$ contains a union is similar to the previous one.


      \item If some $e \in \dom{\ctx_d}$ contains a recursive type or an intersection, we use \cref{algorithmic:property-invertible} on $\FF_1$ and apply the induction hypothesis. The result follows from $\mu\Left$ or $\intersect\Left$, respectively, applied on either $\ctx_d^1$ or $\ctx_d^2$, depending on whichever contains $e$.
      \item If some $e \in \dom{\ctx_d}$ contains a union, we use \cref{algorithmic:property-invertible} on $\FF_1$ and apply the induction hypotheses, which give:
      \begin{mathpar}
        \GG_1 : \typeRecA {\ctx_d^1, e : \parens{\typeListB', B_1}} \emptyset {R_a} a {D_1^1}
        \and \HH_1 : \typeRecA {\ctx_d^2} \emptyset {Q'} d {D_2^1}
        \and \DD_1 : \typeRecA {\ctx_c', a : D_1^1, d : D_2^1} \emptyset {P_a'} c \typeList
        \\ \GG_2 : \typeRecA {\ctx_d^1, e : \parens{\typeListB', B_2}} \emptyset {R_a} a {D_1^2}
        \and \HH_2 : \typeRecA {\ctx_d^2} \emptyset {Q'} d {D_2^2}
        \and \DD_2 : \typeRecA {\ctx_c', a : D_1^2, d : D_2^2} \emptyset {P_a'} c \typeList
      \end{mathpar}
      Here, we assume without loss of generality that $e \in \dom{\ctx_d^1}$. We have also unified the contexts using \cref{refinements:regularity-corollary}. From \cref{algorithmic:typing-weakening} and $\union\Right$ on $\GG_1$ and $\GG_2$, followed by $\union\Left$ on the results, we get:
      $$
        \GG : \typeRecA {\ctx_d^1, e : \typeListB} \emptyset {R_a} a {D_1^1 \union D_1^2}.
      $$
      From $\intersect\Right$ on $\HH_1$ and $\HH_2$, we get
      $$
        \HH : \typeRecA {\ctx_d^2} \emptyset {Q'} d {D_2^1 \intersect D_2^2}.
      $$
      Finally, by \cref{algorithmic:typing-weakening} and $\intersect\Left$ on $\DD_1$ and $\DD_2$, followed by $\union\Left$ on the results, we have
      $$
        \typeRecA {\ctx_c', a : D_1^1 \union D_1^2, d : D_2^1 \intersect D_2^2} \emptyset {P_a'} c \typeList.
      $$
      We pick $D_1 = D_1^1 \union D_1^2$ and $D_2 = D_2^1 \intersect D_2^2$ for the result.

      \item Otherwise, we have $\structural {\ctx_c'}$, $\structural {\ctx_d}$, and $\structural\typeList$, which means \cref{algorithmic:property-invertible-extended-corollary} becomes applicable on $\DD$ and $\FF_1$. We case on the structure of $D$:
        \begin{itemize}
          \item If $D$ is a recursive type, $\DD$ must be by $\mu\Left$ and $\FF_1$ must be by $\mu\Right$. We apply the induction hypotheses on the premises, which immediately gives the result.
          \item If $D = D_1 \intersect D_2$, then inversion on $\FF_1$ gives $\GG_1 : \typeRecA {\ctx_d} \emptyset Q d {D_1}$ and $\GG_2 : \typeRecA {\ctx_d} \emptyset Q d {D_2}$. Additionally, \Cref{algorithmic:property-invertible-extended-corollary} on $\DD$ gives $\HH_1 : \typeRecA {\ctx_c, d : D_1} \emptyset P c \typeList$ or $\HH_2 : \typeRecA {\ctx_c, d : D_2} \emptyset P c \typeList$. Applying induction hypothesis with $\GG_1$ and $\HH_1$ or with $\GG_2$ and $\HH_2$ (depending on which side of the or we have) gives the result.

          \item If $D = D_1 \union D_2$, we apply inversion on $\DD$ and \cref{algorithmic:property-invertible-extended-corollary} on $\FF_1$. We get matching cases, and we can apply the induction hypothesis.

          \item Otherwise, $\structural D$. Inversion on $\FF_1$ gives $D = D_1 \tensor D_2$ and the corresponding typing derivations for $R_x$ and $Q'$. Inversion on $\DD$ gives the typing derivation for $P'_x$. We get the desired result by substituting $a$ for $x$.
        \end{itemize}
    \end{itemize}
  \end{proof}
  Finally, we can split $\config_d$ into $\config_d^1$ and $\config_d^2$ such that $\providesCtx {\config_d^1} {\ctx_d^1}$ and $\providesCtx {\config_d^2} {\ctx_d^2}$. It is simple to put together a derivation for $\config_2$ using the results above.


  \item [\irb{internal}, \irb{lolli}, \irb{external} :] Similar to above.

  \end{description}
\end{proof}

