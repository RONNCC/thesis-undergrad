
\section{Metatheory}

Just like we did for type equivalence, we expect the subtyping relation to satisfy certain properties. Mainly, it should be a preorder, that is, it should satisfy reflexivity and transitivity. The following \namecref{base:sub-is-preorder} shows these are admissible:

\begin{theorem}
  \label{base:sub-is-preorder}
  $\sub$ is a preorder:
  \begin{itemize}
    \item $A \sub A$ for all types $A$.
    \item $A \sub B$ and $B \sub C$ implies $A \sub C$ for all types $A, B, C$.
  \end{itemize}
\end{theorem}
\begin{proof}
  Follows from a simple coinduction for reflexivity. For transitivity, we use a lexicographic combination of coinduction on the two subtyping derivations and induction on $\size{(B)}$. We omit details here since the proof is standard and since we will be switching to a different subtyping relation in \cref{refinements:subtyping}.
\end{proof}


\subsection{Type Safety}

We did not add new forms of processes, so reduction and the notion of being poised is the same as in \cref{base:semantics}. We have, however, replaced type equality with a more general notion of subtyping, so we need to revisit the progress and preservation theorems.

The proofs of progress and preservation theorems are slight modifications of the ones we presented in \cref{base:semantics}: we simply replace every occurrence of $\typeeq$ with $\sub$. It should be straightforward to modify the process inversion \namecref{inversion-poised-eq} (\cref{inversion-poised-eq}) which is used in the proof of the progress theorem, and the two cases related to $\typeeq$, so we omit details here.

