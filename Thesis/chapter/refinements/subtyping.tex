
\section{Subtyping Revisited}
\label{refinements:subtyping}

Every refinement system requires a notion of subtyping to be practical since subtyping is needed in order to implicitly propagate refinements. For example, if we have a process providing $\pos$, we should be able to use it in a context that requires $\nat$ since $\pos$ is a more specific type. Otherwise, we would require explicit coercions to erase extra information which can easily become cumbersome, especially when we have multiple refinements on a type and we need a specific subset. However, addition of non-structural types such as intersection and union complicates subtyping since these types do not depend on or reveal the structure of the processes they describe. More specifically, property types raise questions about the soundness of completeness of the subtyping relation.

Subtyping is said to be sound if whenever $A \sub B$, using processes of type $A$ in contexts expecting a process of type $B$ does not break type safety. This is usually requires (1) all terms of types $A$ and $B$ to have the same structure, and (2) the set of possible behaviors of terms with type $A$ to be a subset of the set of possible behaviors of terms with type $B$. Some systems, mainly the ones that use coercive subtyping \todo{References to coercive subtyping}, may not necessarily require condition (1). We will in this system though, since we do not want term level constructs that witness convertibility. Subtyping is complete if every time it is safe to use $A$ for $B$, we have $A \sub B$.

Soundness of subtyping is necessary for type safety, so we have no choice but to make sure this is the case. On the other hand, type safety will hold even in the presence of incompleteness. In fact, most practical systems give up on completeness since it usually turns out to be very hard to design a simple and complete subtyping relation. This will be the case for our system, though, we do try to find a middle ground between simplicity and completeness in order to recover some rules we believe are important in practice.

\todo{Explain why completeness is important?}

\subsection{Distributivity Laws}
\label{distributivity}

The subtyping relation we give in \cref{base:subtyping} and in the previous sections is not complete with respect to, say, the ideal semantics of types \cite{VouillonM04, Damm94}. This is because intersections and unions admit many distributivity-like rules over structural types and over each other. For example, it is not hard to see that $(A_1 \tensor A_2) \intersect (B_1 \tensor B_2) \sub (A_1 \intersect B_1) \tensor (A_2 \intersect B_2)$ would be sound using a propositional reading: if a process sends out a channel that satisfies $A_1$ then acts as $B_1$, \emph{and} the sent channel also satisfies $A_2$ in addition to the result satisfying $B_2$, then the channel satisfies both $A_1$ and $A_2$ and the result satisfies both $B_1$ and $B_2$. However, this judgement is not admissible in the given system: the only applicable rules are $\Sub{\intersect\Left_1}$ and $\Sub{\intersect\Left_2}$, both of which get stuck because we lose half the information we require for the rest of the derivation. The situation is perhaps exacerbated by the fact that we \emph{can} prove subtyping in the other direction, so these types are supposed to be equivalent. This means depending on where these types occur, we may fail to prove one side of a symmetric relation!

The fix is not as simple as adding this rule as an extra axiom. For one, it is not trivial to rewrite this rule in order to preserve admissibility of transitivity. More importantly, this is not the only rule we would have to add. \Cref{distributivity-examples} gives just \emph{some} of the many sound rules that are not admissible.

\begin{rules}[distributivity-examples]{Sound but inadmissible subtyping rules}
   % Intersection
   (A_1 \tensor A_2) \intersect (B_1 \tensor B_2) \sub (A_1 \intersect B_1) \tensor (A_2 \intersect B_2) \\
   \internals A I \intersect \internals B J \sub \internal\braces{\lab_x : A_x \intersect B_x}_{x \in I \cap J} \\
   (A \lolli B_1) \intersect (A \lolli B_2) \sub A \lolli (B_1 \intersect B_2) \\
   \externals A I \intersect \externals B J \sub \externals A I \cup \externals B J ~ (I \cap J = \emptyset)\\
   % Union
   (A_1 \union A_2) \tensor B \sub (A_1 \tensor B) \union (A_2 \tensor B) \\
   \internals A I \cup \internals B J \sub \internals A I \union \internals B J ~ (I \cap J = \emptyset) \\
   (A_1 \lolli B) \intersect (A_2 \lolli B) \sub (A_1 \union A_2) \lolli B \\
   \external\braces{\lab_x : A_x \union B_x}_{x \in I \cap J} \sub \externals A I \union \externals B J \\
   % Intersection and union
   (A_1 \union B) \intersect (A_2 \union B) \sub (A_1 \intersect A_2) \union B \\
   (A_1 \union A_2) \intersect B \sub (A_1 \intersect B) \union (A_2 \intersect B)
\end{rules}

This list is certainly not complete, and there are almost as many rules in this list as there are in the original system. Clearly, a blind axiomatic approach which adds all admissible rules is not practical and a more general treatment is in order. There has been some work in incorporating intersections and unions in a conventional type system that preserves completeness under certain conditions. The closest and most complete system we found was from Damm \cite{Damm94, Damm94p2}. In \cite{Damm94}, he encodes types as regular tree expressions, and reformulates subtyping as regular tree grammar containment which was shown to be decidable. This results in a system that is sound and complete when all types are infinite (but $\terminate$ is a finite type). \cite{Damm94p2} extends this work such that the system is sound in the presence of finite types (although not necessarily complete).

Their system is very close to what we would like to accomplish, however, we think it is too complicated for our purposes as it requires familiarity with ideal semantics of types (which in turn is based on domain theory and the theory metric spaces) and regular tree grammars. Even if a similar approach could be made to work, we find such systems to be fragile in the face of future extensions. We would rather work with a simple and robust subtyping relation that does not necessitate rethinking every detail with every extension, so we make a design decision: we give up full completeness and instead design a system that admits the rules we are likely to encounter in practice.

In the next \namecref{refinements:subtyping-multi}, we present a system that we think achieves the right tradeoff between simplicity and completeness.


\subsection{Subtyping as Sequent Calculus with Multiple Conclusions}
\label{refinements:subtyping-multi}

Since we are interested in intersections and unions, we would like to at least admit distributivity of intersection over union and vice versa. That is, we would like the following equalities to hold:
\begin{mathpar}
   (A_1 \union B) \intersect (A_2 \union B) \typeeq (A_1 \intersect A_2) \union B \\
   (A_1 \union A_2) \intersect B \typeeq (A_1 \intersect B) \union (A_2 \intersect B)
\end{mathpar}
Going from right to left turns out to be easy, but we quickly run into a problem if we try to do the other direction: whether we break down the union on the right or the intersection on the left, we always lose half the information we need to carry out the rest of the proof.%
\footnote{This issue does not come up in the other direction since intersection right and union left rules are invertible, that is, they preserve all information.}
\todo{Are these paragraphs repetitive?}

Our solution is doing the obvious: if the problem is losing half the information, well, we should just keep it around. This suggests a system where the single type on the left and the type on right are replaced with \emph{(multi)sets} of types. That is, instead of the judgment $A \le B$, we use a judgment of the form $A_1, \ldots, A_n \subA B_1, \ldots, B_n$, where the left of $\subA$ is interpreted as a conjunction (intersection) and the right is interpreted as a disjunction (union). This results in a system reminiscent of classical sequent calculus with multiple conclusions \cite{Gentzen35, Girard87}. However, our system is slightly different since we are working with coinductive rules.

The new system is Figure~\ref{refinements:subtyping-multi-rules}. We use $\typeList$ and $\typeListB$ to denote multisets of types. The intersection left rules are combined into one rule that keeps both branches around. The same is done with union right rules. Intersection right and union left rules split into two derivations, one for each branch, but keep the rest of the types unchanged. We can unfold a recursive type on the left or on the right. When we choose to apply a structural rule, we have to pick exactly one type on the left and one on the right with the same structure. This results is a system that does not admit distributivity of intersections and unions over structural types. We conjecture that matching multiple types will give distributivity over structural types, though the intricacies that arise has lead us to leave this to future work.

\begin{rules}[refinements:subtyping-multi-rules]{Subtyping with multiple hypothesis and conclusions}
  % Intersection
  \infer=[\SubA{\intersect\Right}]{\typeList \subA A_1 \intersect A_2, \typeListB}
    {\typeList \subA A_1, \typeListB \and \typeList \subA A_2, \typeListB}
  \and \infer=[\SubA{\intersect}\Left]{\typeList, A_1 \intersect A_2 \subA \typeListB}
    {\typeList, A_1, A_2 \subA \typeListB}
  % Union
  \\ \infer=[\SubA{\union\Right}]{\typeList \subA A_1 \union A_2, \typeListB}
    {\typeList \subA A_1, A_2, \typeListB}
  \and \infer=[\SubA{\union}\Left]{\typeList, A_1 \union A_2 \subA \typeListB}
    {\typeList, A_1 \subA \typeListB & \typeList, A_2 \subA \typeListB}
  % Structural
  \\ \infer=[\SubA{\terminate}]{\typeList, \terminate \subA \terminate, \typeListB}{}
  \and \infer=[\SubA\tensor]{\typeList, A \tensor B \subA A' \tensor B', \typeListB}
    {A \subA A' & B \subA B'}
  \and \infer=[\SubA\internal]{\typeList, \internals A I \subA \internals {A'} J, \typeListB}
    { I \subseteq J
    & A_\indexVar \subA A_\indexVar'~\text{for}~\indexVar\in I
    }
  \and \infer=[\SubA\lolli]{\typeList, A \lolli B \subA A' \lolli B', \typeListB}
    {A' \subA A & B \subA B'}
  \and \infer=[\SubA\external]{\typeList, \externals A I \subA \externals {A'} J, \typeListB}
    { J \subseteq I
    & A_\indexVar \subA A_\indexVar'~\text{for}~\indexVar\in J
    }
  % Recursive
  \\ \infer=[\SubA{\mu\Right}]{\typeList \subA \recursive t A, \typeListB}
     {\typeList \subA \subst {\recursive t A} t A, \typeListB}
  \and \infer=[\SubA{\mu\Left}]{\typeList, \recursive t A \subA \typeListB}
     {\typeList, \subst {\recursive t A} t A \subA \typeListB}
\end{rules}

Subtyping of singular types like we had before is defined as expected: $A \sub B$ if and only if $A \subA B$. \todo{Do I want to say this?}


\subsection{Properties}

First, let us make sure that the new system in fact admits the desired distributivity rules.
\todo{Show the derivations of distributivity as an example.}

Next, we verify that usual properties expected of logical systems (such as weakening and cut admissibility) are valid for our system.

\begin{lemma}[Weakening]
  If $\typeList \subA \typeListB$, then  $\typeList, \typeList' \subA \typeListB, \typeListB'$ for all $\typeList', \typeListB'$.
\end{lemma}
\begin{proof}
  Follows from a trivial coinduction on the derivation of $\typeList \subA \typeListB$ since all rules are parametric on the unused types.
\end{proof}

\begin{lemma}[Admissibility of Identity]
  $\typeList \subA \typeList$ for all non-empty $\typeList$.
\end{lemma}
\begin{proof}
  Assume \todo{Resolve the same problem for equality and old subtyping first.}
\end{proof}

\begin{lemma}[Admissibility of Cut]
  If $\typeList \subA \typeList, A$ and $\typeList, A \subA \typeListB$, then $\typeList \subA \typeListB$.
\end{lemma}
\begin{proof}
  
\end{proof}


We should check that the new subtyping relation is a preorder:
\begin{theorem}
  \label{refinements:sub-is-preorder}
  $\sub$ is a preorder:
  \begin{itemize}
    \item $A \sub A$ for all types $A$.
    \item $A \sub B$ and $B \sub C$ implies $A \sub C$ for all types $A, B, C$.
  \end{itemize}
\end{theorem}

