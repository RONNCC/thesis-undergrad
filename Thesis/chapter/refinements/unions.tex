
\section{Union Types}

Unions are represented by the following syntactic construct:
\begin{center}
\begin{tabular}{l c l l}
  $A, B, C$ & ::= & \ldots                & everything from before \\
            & $|$   & $A \union B$        & act as either $A$ or $B$
\end{tabular}
\end{center}

Unions are the dual of intersections, which means they correspond to logical disjunction. Unions express the property that a process behaves as either $A$ or $B.$ Being the dual to intersections, the right rules are derivable by subsumption.

\begin{mathpar}
  \parens*{\infer[\union\Right_1]{\typeDJ{P}{c}{A \union B}}
    {\typeDJ{P}{c}{A}}}
  \and \parens*{\infer[\union\Right_2]{\typeDJ{P}{c}{A \union B}}
    {\typeDJ{P}{c}{B}}}
  \and \infer[\union\Left]{\typeD{\ctx, c : A \union B}{P}{d}{D}}
    {\typeD{\ctx, c : A}{P}{d}{D} & \typeD{\ctx, c : B}{P}{d}{D}}
\end{mathpar}

Unions have the following (algorithmic) subtyping rules:
\begin{mathpar}
  \infer[\Sub{\union}\Right_1]{A \sub B_1 \union B_2}
    {A \sub B_1}
  \and \infer[\Sub{\union}\Right_1]{A \sub B_1 \union B_2}
    {A \sub B_2}
  \and \infer[\Sub{\union}\Left]{A_1 \union A_2 \sub B}
    {A_1 \sub B & A_2 \sub B}
\end{mathpar}

We again leave out the distributivity rules.

