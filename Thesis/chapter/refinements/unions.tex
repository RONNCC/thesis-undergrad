
\section{Union Types}

Unions are the dual of intersections and correspond to processes that satisfy one or the other property, and are written $A \union B$. We add unions because they are a natural extension to a  type system with intersections. We will also see how $n$-ary internal choice can be interpreted as
the union of singleton choices. Without them, our interpretation would only be half-complete since we could interpret external choice (with intersections) but not internal choice.

Being dual to intersections, the typing rules for unions mirror the typing rules for intersections: we have two right rules and one left rule, and this time the right rules are derivable from subtyping. The rules are given below:

\begin{mathpar}
  \infer[\union\Right_1]{\typeRecDJ{P}{c}{A \union B}}
    {\typeRecDJ{P}{c}{A}}
  \and \infer[\union\Right_2]{\typeRecDJ{P}{c}{A \union B}}
    {\typeRecDJ{P}{c}{B}}
  \and \infer[\union\Left]{\typeRecD{\ctx, c : A \union B}{\recCtx}{P}{d}{D}}
    {\typeRecD{\ctx, c : A}{\recCtx}{P}{d}{D} & \typeRecD{\ctx, c : B}{\recCtx}{P}{d}{D}}
\end{mathpar}

The right rules state the process has to offer either the left type or the right type respectively. The left rule says we need to be prepared to handle either type. It is important to point out that we restore a long-lost symmetry for functional languages. The natural left rule we give here for unions (natural since it is dual to the right rule for intersection) has been shown to be problematic in functional languages \cite{Barbanera95ic}. One solution limits the left rule to expressions in evaluation position \cite{DunfieldP04}. The straightforward left rule turns out to be already sound here due to our use of the linear sequent calculus.

The usual subtyping rules are given below. Dual to intersections, the right typing rules become redundant with the addition of the subtyping rules.
\begin{mathpar}
  \infer=[\Sub{\union\Right_1}]{A \sub B_1 \union B_2}
    {A \sub B_1}
  \and \infer=[\Sub{\union\Right_1}]{A \sub B_1 \union B_2}
    {A \sub B_2}
  \and \infer=[\Sub{\union\Left}]{A_1 \union A_2 \sub B}
    {A_1 \sub B & A_2 \sub B}
\end{mathpar}

Unions allow us to describe some interesting properties. For example, we can show that every natural is either even or odd:
\begin{lstlisting}[language=krill, style=custom]
  iso : Nat -o (Even or Odd)
  `c <- iso `d =
    case `d of
      zero -> wait `d; `c.zero; close `c
      succ -> `c.succ; `e <- iso `d; `c <- `e
\end{lstlisting}
We have to unfold one level since our system cannot prove $Nat \sub Even \union Odd$. A more involved example is given in \cref{example:bitstrings}.

Definitions of unguarded variables and $\size$ are extended similarly:
\begin{align*}
  \unguarded{(A \union B)} &= \unguarded{(A)} \cup \unguarded{(B)} \\
  \size{(A \union B)} &= 1 + \size{(A)} + \size{(B)}
\end{align*}

