
\section{Intersection Types}

We denote the intersection of two types $A$ and $B$ as $A \intersect B$. A process offers an intersection type if its behavior satisfies both types simultaneously. Using intersections, we can assign the programs introduced in \cref{process-expressions} types specifying all behavioral properties we care about:
\begin{lstlisting}[language=krill, style=custom]
  z : Nat and Even
  s : (Nat -o Nat) and (Even -o Odd) and (Odd -o Even)
  double : (Nat -o Nat) and (Nat -o Even)
\end{lstlisting}

Note that as is usual with intersections, multiple types are assigned to \emph{the same process}. Put differently, we cannot use two different processes or specify two different behaviors to satisfy the different branches of an intersection. This leads to the following typing rule:
$$
  \infer[\intersect\Right]{\typeRecDJ{P}{c}{A \intersect B}}
    {\typeRecDJ{P}{c}{A} & \typeRecDJ{P}{c}{B}}
$$

When we are using a channel on the left that offers an intersection of two types, we know it has to satisfy both properties so we get to pick the one we want:
\begin{mathpar}
  \infer[\intersect\Left_1]{\typeRecD{\ctx, c : A \intersect B}{\recCtx}{P}{d}{D}}
    {\typeRecD{\ctx, c : A}{\recCtx}{P}{d}{D}}
  \and \infer[\intersect\Left_2]{\typeRecD{\ctx, c : A \intersect B}{\recCtx}{P}{d}{D}}
    {\typeRecD{\ctx, c : B}{\recCtx}{P}{d}{D}}
\end{mathpar}

The standard subtyping rules for intersections are given below. It should be noted that the left typing rules above are derivable by an application of subsumption on the left using $\Sub{\intersect\Left_1}$ and $\Sub{\intersect\Left_2}$, so we will not explicitly add these to the final system. Also, we modify the subtyping relation in \cref{refinements:subtyping}, so these rules are temporary. \todo{Maybe delete this last sentence?}

\begin{mathpar}
  \infer=[\Sub{\intersect}\Right]{A \sub B_1 \intersect B_2}
    {A \sub B_1 \and A \sub B_2}
  \and \infer=[\Sub{\intersect\Left_1}]{A_1 \intersect A_2 \sub B}
    {A_1 \sub B}
  \and \infer=[\Sub{\intersect\Left_2}]{A_1 \intersect A_2 \sub B}
    {A_2 \sub B}
\end{mathpar}

Since we are extending the language of types, we need to revisit contractiveness. Since they do not have a corresponding expression level construct, intersections are not considered structural types. Thus, they propagate unguarded variables:
\begin{align*}
  \unguarded{(A \intersect B)} &= \unguarded{(A)} \cup \unguarded{(B)} \\
  \size{(A \intersect B)} &= \size{(A)} + \size{(B)}
\end{align*}

