
\section{Type Safety}
\label{refinements:semantics}

As usual, we need to prove type safety for the extended system. Note that we did not add new forms of processes, so reduction and the notion of being poised is the same as in \cref{base:semantics}. We have, however, altered typing by replacing type equality with subtyping, and by adding intersections and unions. This necessitates reproving some of the helper lemmas and modifying the proofs of progress and preservation theorems slightly.


\subsection{Progress}

First, we need to change the inversion lemma for process typing to use subtyping rather than type equality:

\begin{lemma}[Inversion of Process Typing]
\label{refinements:inversion-poised-eq}
  If $\typeRecD \ctx \emptyset P c A$ and $\poised{\proc c P}$, then:
  \begin{itemize}
    \item If $A \sub \terminate$ then $P = \tclose c$.
    \item If $A \sub A_1 \tensor A_2$ then $P = \tsend c d {Q_d} {P'}$.
    \item If $A \sub \internals A I$ then $P = \tselect c x {P'}$ where $x \in I$.
    \item If $A \sub A_1 \lolli A_2$ then $P = \trecv d c {P'}$.
    \item If $A \sub \externals A I$ then $P = \tcase c {\tbranches {P'} J}$ where $I \subseteq J$.
  \end{itemize}
\end{lemma}

\begin{proof}
  By induction on the derivation of $\typeRecD \ctx \emptyset P c A$ followed by inversion on $\poised{\proc c P}$ or the subtyping judgement for impossible cases. \todo{Do the inversion proof.}
\end{proof}


The statement of the progress lemma does not change:

\begin{theorem}[Progress]
If $\providesCtx \config \ctx$ then either
\begin{enumerate}
  \item $\steps{\config}{\config'}$ for some $\config'$, or
  \item $\config$ is poised.
\end{enumerate}
\end{theorem}

\begin{proof}
We will extend the proof of \cref{base:progress}. Again, we will use induction on configuration typing followed by a nested induction on the typing derivation in the single channel case. The case for multiple channels does not change. For the single channel case, we know $\config = \config', \proc c P$, and by inversion, $\typeRecD \ctx \emptyset P c A$ and $\providesCtx {\config'} \ctx$. Assume $\config'$ is poised by induction.

  Define $\pred{\typeRecD{\ctx'}{\emptyset}{P}{c}{A'}} :=$ if $\providesCtx {\config'} {\ctx'}$ then either $\proc c P$ is poised or $\config$ can take a step. We proceed by induction on the typing derivation.
    \begin{description}
      \item[Case $\id, \cut, \terminate\Right, \tensor\Right, \internal\Right, \lolli\Right, \external\Right, \rec :$] Same as before.

      \item[Case $\terminate\Left, \tensor\Left, \internal\Left, \lolli\Left, \external\Left :$] Same as before, but using \cref{refinements:inversion-poised-eq} instead of \cref{inversion-poised-eq}.

      \item[Case $\intersect\Right, \union\Right, \irb{Sub}\Right :$] Follows immediately from the induction hypothesis (using either branch for $\intersect\Right$).

      \item[Case $\intersect\Left, \union\Left, \irb{Sub}\Left :$] In all cases, $\ctx' = \ctx'', d : D$ and $\typeRecD{\ctx'', d : D'}{\emptyset}{P}{c}{A'}$ for some $D'$ where $D \sub D'$%
      \footnote{$D'$ is one of the branches of $\intersect$ or $\union$ in the relevant cases. The branch is predetermined for  $\intersect\Left$. For $\union\Left$, either suffices.}

      \item[Case $\union\Left :$]

      \item[Case $\irb{Sub}\Left :$] $\ctx' = \ctx'', d : D$ and $\typeRecD{\ctx'', d : D'}{\emptyset}{P}{c}{A'}$ for some $D'$ where $D \sub D'$. Then, $\ctx' \sub \ctx'', d : D'$, so the result follows from \cref{refinements:sub-configuration} and the induction hypothesis.

    \end{description}

    Finally, we get either $\poised{\proc c P}$ or $\config$ steps from $\pred{\typeRecD \ctx \emptyset P c A}$ where $\ctx \sub \ctx$ by \cref{sub-is-preorder}. In the former case, $\config$ is poised since $\config'$ is poised from before, and in the latter case we are immediately done.
\end{proof}


\subsection{Preservation}

