
\section{Process Expressions}
\label{base:process-expressions}

Within this framework, proof terms correspond to processes. For example, cut, written $\tspawn{c}{P_c}{Q_c}$, denotes a form of process composition where the client spawns off a helper process ($P_c$) with which it can communicate from then on. The intuition is formalized in the typing rule:
$$ \infer[\cut]{ \typeD {\ctx, \ctx'} {\tspawn{c}{P_c}{Q_c}} {d} {D} }
    { \typeD {\ctx} {P_c} {c} {A}
    & \typeD {\ctx', c : A} {Q_c} {d} {D}
    }
$$

The rest of the process expressions are summarized below, with the sending construct
followed by the matching receiving construct. Discussion of other typing rules is given in \cref{base:type-assignment}.

\begin{center}
\begin{tabular}{l c l l}
  $P, Q, R$ & ::= & $\tspawn{x}{P_x}{Q_x}$     & cut (spawn) \\
            & $|$ & $\tfwd c d$                & id (forward) \\
            & $|$ & $\tclose c \mid \twait c P$  & $\terminate$ \\
            & $|$ & $\tsend{c}{y}{P_y}{Q} \mid \trecv{x}{c}{R_x}$ & $A \tensor B,$ $A \lolli B$ \\
            & $|$ & $\tselect{c}{}{P} \mid \tcase c {\tbranches Q i}$  & $\externals A I,$ $\internals A I$ \\
            & $|$ & $\tsendVal{c}{M}{Q} \mid \trecvVal{n}{c}{R_n}$ & $A \sendVal B,$ $A \recvVal B$
\end{tabular}
\end{center}

Note that cut, $\mathtt{send}$, and $\mathtt{recv}$ bind the spawned, sent, and received channel names, which means these are identified up to alpha conversion. We denote the usual capture avoiding simultaneous substitution of channels $\tvector c$ for $\tvector x$ in $P$ by $\subst {\tvector c} {\tvector x} P$ where $\tvector c$ and $\tvector x$ are ordered sequences of equal length.

Let us consider an example program to get more intuition about the system. We will use process level natural numbers, $\nat$, as a running example. Note that we use concrete syntax to write programs, but the mapping to abstract syntax presented above should be clear. Also, this example (and almost any interesting one) requires recursive types, which are introduced in \cref{base:recursion}.

First, we define the interface:

\begin{lstlisting}[language=krill, style=custom]
  type Nat = +{zero : 1, succ : Nat}
\end{lstlisting}

The interface states that a process level natural number is an internal choice of either zero or a successor of another natural. Next, we define two simple processes that implement the interface:

\begin{minipage}{.48\textwidth}
\begin{lstlisting}[language=krill, style=custom]
  z : Nat
  `c <- z =
    `c.zero;
    close `c
\end{lstlisting}
\end{minipage}
\hfill
\begin{minipage}{.48\textwidth}
\begin{lstlisting}[language=krill, style=custom]
  s : Nat -o Nat
  `c <- s `d =
    `c.succ;
    `c <- `d
\end{lstlisting}
\end{minipage}

\texttt{z} simply sends the label \texttt{zero} along the channel \texttt{`c} (which it provides) and terminates, whereas \texttt{s} send the label \texttt{succ} and delegates to \texttt{`d}. Here is a slightly more complicated example that uses recursion:

\begin{lstlisting}[language=krill, style=custom]
  double : Nat -o Nat
  `c <- double `d =
    case `d of
      zero -> `c.zero; wait `d; close `c
      succ -> `c.succ; `c.succ; `c <- double `d
\end{lstlisting}


\subsection{Type Assignment for Processes}
\label{base:type-assignment}

The typing rules for other constructs are derived from linear logic by decorating derivations with proof terms. The rules are given in \cref{session-assignment}. Note that we allow unused branches case expressions for $\internal\Left$ and $\external\Right$, which makes width subtyping easier (discussed in \cref{base:subtyping}).

\begin{rules}[session-assignment]{Type assignment to processes}
  % id and cut
  \infer[\id]{ \typeD {c : A} {\tfwd{d}{c}} {d} {A} }
    {}
  \and \infer[\cut]{ \typeD {\ctx, \ctx'} {\tspawn{c}{P_c}{Q_c}} {d} {D} }
    { \typeD {\ctx} {P_c} {c} {A}
    & \typeD {\ctx', c : A} {Q_c} {d} {D}
    }
  % Terminate
  \and \infer[\terminate\Right]{\typeD{\emptyCtx}{\tclose c}{c}{\terminate}}
    {}
  \and \infer[\terminate\Left]{\typeD{\ctx, c : \terminate}{\twait c P}{d}{A}}
    {\typeDJ{P}{d}{A}}
  % Tensor
  \and \infer[\tensor\Right]{\typeD{\ctx, \ctx'}{\tsend{c}{d}{P_d}{Q}}{c}{A \tensor B}}
    { \typeD{\ctx}{P}{d}{A}
    & \typeD{\ctx'}{Q}{c}{B}
    }
  \and \infer[\tensor\Left]{ \typeD{\ctx, c : A \tensor B}{\trecv{d}{c}{P_d}}{e}{E} }
    { \typeD{\ctx, d : A, c : B}{P_d}{e}{E} }
  % Internal choice
  \and \infer[\internal\Right]{\typeDJ { \tselect{c}{i}{P} } {c} {\internals{A}{I}} }
    { i \in I
    & \typeDJ{P}{c}{A_i}
    }
  \and \infer[\internal\Left]{ \typeD { \ctx, c : \internals{A}{I} } { \tcase{c}{\tbranches{P}{J}} } {d} {D} }
   { I \subseteq J
   & \typeD{\ctx, c : A_k}{P_k}{d}{D}~\text{for}~k\in I
   }
  % Implication
  \and \infer[\lolli\Right]{ \typeD{\ctx}{\trecv{d}{c}{P_d}}{c}{A \lolli B} }
    { \typeD{\ctx, d : A}{P_d}{c}{B} }
  \and \infer[\lolli\Left]{\typeD{\ctx, \ctx', c : A \lolli B}{ \tsend{c}{d}{P_d}{Q} } {e}{E}}
    { \typeD{\ctx}{P_d}{d}{A}
    & \typeD{\ctx', c : B}{Q}{e}{E}
    }
  % External choice
  \and \infer[\external\Right]{ \typeDJ { \tcase{c}{\tbranches{P}{I}} } {c} {\externals{A}{J}} }
   { J \subseteq I
   & \typeDJ{P_k}{c}{A_k}~\text{for}~k\in J
   }
  \and \infer[\external\Left]{\typeD{\ctx, c : \externals{A}{I}} { \tselect{c}{i}{P} } {d} {D}}
    { i \in I
    & \typeD{\ctx, c : A_i}{P}{d}{D}
    }
\end{rules}

As usual, we identify bound channels up to alpha conversion. Free channels are subject to consistent renaming across a sequent by substitution.
%We can change free channel names while preserving typing as formalized below:
% \begin{lemma}[Channel Substitution]
%   \label{channel-substitution-pre}
%   If $\typeDJ P c A$ and $\tvector z \cap \free {\parens P} = \emptyset$ then $\typeD {\subst {\tvector z} {\tvector y} \ctx} {\subst {\tvector z} {\tvector y} P} {\subst {\tvector z} {\tvector y} c} A$.
% \end{lemma}

% We delay the proof until after we introduce recursive types since the typing-derivation will be changed slightly.


