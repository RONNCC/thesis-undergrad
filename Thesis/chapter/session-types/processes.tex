
\section{Process Expressions}

Below is a summary of the process expressions, with the sending construct
followed by the matching receiving construct.

\begin{center}
\begin{tabular}{l c l l}
  $P, Q, R$ & ::= & $\tspawn{x}{P_x}{Q_x}$     & cut (spawn) \\
            & $|$ & $\tfwd c d$                & id (forward) \\
            & $|$ & $\tclose c \mid \twait c P$  & $\terminate$ \\
            & $|$ & $\tsend{c}{y}{P_y}{Q} \mid \trecv{x}{c}{R_x}$ & $A \tensor B,$ $A \lolli B$ \\
            & $|$ & $\tselect{c}{}{P} \mid \tcase c {\tbranches Q i}$  & $\externals A I,$ $\internals A I$ \\
            & $|$ & $\tsendVal{c}{M}{Q} \mid \trecvVal{n}{c}{R_n}$ & $A \sendVal B,$ $A \recvVal B$
\end{tabular}
\end{center}


\subsection{Type Assignment for Processes}

We assign sessions to processes using the rules in \cref{session-assignment}.

\begin{rules}[session-assignment]{Process assignment}
  % id and cut
  \infer[\id]{ \typeD {c : A} {\tfwd{d}{c}} {d} {A} }
    {}
  \and \infer[\cut]{ \typeD {\ctx, \ctx'} {\tspawn{c}{P_c}{Q_c}} {d} {D} }
    { \typeD {\ctx} {P_c} {c} {A}
    & \typeD {\ctx', c : A} {Q_c} {d} {D}
    }
  % Terminate
  \and \infer[\terminate\Right]{\typeD{\emptyCtx}{\tclose c}{c}{\terminate}}
    {}
  \and \infer[\terminate\Left]{\typeD{\ctx, c : \terminate}{\twait c P}{d}{A}}
    {\typeDJ{P}{d}{A}}
  % Tensor
  \and \infer[\tensor\Right]{\typeD{\ctx, \ctx'}{\tsend{c}{d}{P_d}{Q}}{c}{A \tensor B}}
    { \typeD{\ctx}{P}{d}{A}
    & \typeD{\ctx'}{Q}{c}{B}
    }
  \and \infer[\tensor\Left]{ \typeD{\ctx, c : A \tensor B}{\trecv{d}{c}{P_d}}{e}{E} }
    { \typeD{\ctx, d : A, c : B}{P_d}{e}{E} }
  % Internal choice
  \and \infer[\internal\Right]{\typeDJ { \tselect{c}{i}{P} } {c} {\internals{A}{I}} }
    { i \in I
    & \typeDJ{P}{c}{A_i}
    }
  \and \parens*{\infer[\internal\Right]{\typeDJ { \tselect{c}{}{P} } {c} {\internal \braces{\lab : A}} }
    { \typeDJ{P}{c}{A}
    }
  }
  \and \infer[\internal\Left]{ \typeD { \ctx, c : \internals{A}{I} } { \tcase{c}{\tbranches{P}{J}} } {d} {D} }
   { I \subseteq J
   & \typeD{\ctx, c : A_k}{P_k}{d}{D}~\text{for}~k\in I
   }
  % Implication
  \and \infer[\lolli\Right]{ \typeD{\ctx}{\trecv{d}{c}{P_d}}{c}{A \lolli B} }
    { \typeD{\ctx, d : A}{P_d}{c}{B} }
  \and \infer[\lolli\Left]{\typeD{\ctx, \ctx', c : A \lolli B}{ \tsend{c}{d}{P_d}{Q} } {e}{E}}
    { \typeD{\ctx}{P_d}{d}{A}
    & \typeD{\ctx', c : B}{Q}{e}{E}
    }
  % External choice
  \and \infer[\external\Right]{ \typeDJ { \tcase{c}{\tbranches{P}{I}} } {c} {\externals{A}{J}} }
   { J \subseteq I
   & \typeDJ{P_k}{c}{A_k}~\text{for}~k\in J
   }
  \and \infer[\external\Left]{\typeD{\ctx, c : \externals{A}{I}} { \tselect{c}{i}{P} } {d} {D}}
    { i \in I
    & \typeD{\ctx, c : A_i}{P}{d}{D}
    }
\end{rules}

\todo{Note that we can simplify $\internal\Right,$ $\internal\Left,$ $\external\Right,$ $\external\Left$ using subtyping.}

As usual, the specific channel names do not matter, that is, the rules respect alpha equivalence. \todo{Is this related to alpha-equivalence? We are renaming free channels not bound ones. What is the right word?} This is formalized by the fact that We can change free channel names while preserving typing as formalized below:
\begin{lemma}[Channel Substitution]
  \label{channel-substitution-pre}
  If $\typeDJ P c A$ and $\tvector z \cap \free {\parens P} = \emptyset$ then $\typeD {\subst {\tvector z} {\tvector y} \ctx} {\subst {\tvector z} {\tvector y} P} {\subst {\tvector z} {\tvector y} c} A$.
\end{lemma}

We delay the proof until after we introduce recursive types since the typing-derivation will be changed slightly.


