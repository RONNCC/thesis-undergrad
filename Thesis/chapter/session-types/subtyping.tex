
\section{Subtyping}

The main purpose of subtyping is to restrict internal choice, relax external choice, and to deal with intersections and unions. Subtyping rules are given in \cref{session-subtyping}.

\begin{rules}[session-subtyping]{Subtyping rules for session types}
  \infer[\Sub{\terminate}]{\terminate \sub \terminate}{}
  \and \infer[\Sub\tensor]{A \tensor B \sub A' \tensor B'}
    {A \sub A' & B \sub B'}
  \and \infer[\Sub\internal]{\internals{A}{I} \sub \internals{A'}{J}}
    {I \subseteq J \and A_k \sub A'_k~\text{for}~k \in J}
  \and \infer[\Sub\lolli]{A \lolli B \sub A' \lolli B'}
    {A' \sub A & B \sub B'}
  \and \infer[\Sub\external]{\externals{A}{I} \sub \externals{A'}{J}}
    {J \subseteq I \and A_k \sub A'_k~\text{for}~k \in J}
\end{rules}

We make use of subtyping with the subsumption rules:

\begin{mathpar}
  \infer[\irb{Sub}\Right]{\typeDJ{P}{c}{A'}}
    {\typeDJ{P}{c}{A} & A \sub A'}
  \and \infer[\irb{Sub}\Left]{\typeD{\ctx, c : A}{P}{d}{B}}
    {\typeD{\ctx, c : A'}{P}{d}{B} & A \sub A'}
\end{mathpar}

Subtyping extends to contexts in the obvious way: $\ctx \sub \ctx'$ if and only if $\ctx = (c_1 : A_1, \ldots, c_n : A_n),$ $\ctx' = (c_1 : A_1', \ldots, c_n : A_n'),$ and $A_i \sub A_i'$ for $1 \le i \le n.$

We require the subtyping relation to be a preorder, that is, it should satisfy reflexivity and transitivity. Since we want an algorithmic system, we do not want to add these as rules. Thus, we have the following admissibility theorem:

\begin{theorem}
  $\sub$ is an equivalence relation:
  \begin{itemize}
    \item $A \sub A$ for all types $A.$
    \item $A \sub B$ and $B \sub C$ implies $A \sub C$ for all types $A, B, C.$
  \end{itemize}
\end{theorem}

