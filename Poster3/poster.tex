%%%%%%%%%%%%%%%%%%%%%%%%%%%%%%%%%%%%%%%%%
% baposter Landscape Poster
% LaTeX Template
% Version 1.0 (11/06/13)
%
% baposter Class Created by:
% Brian Amberg (baposter@brian-amberg.de)
%
% This template has been downloaded from:
% http://www.LaTeXTemplates.com
%
% License:
% CC BY-NC-SA 3.0 (http://creativecommons.org/licenses/by-nc-sa/3.0/)
%
%%%%%%%%%%%%%%%%%%%%%%%%%%%%%%%%%%%%%%%%%

%----------------------------------------------------------------------------------------
%	PACKAGES AND OTHER DOCUMENT CONFIGURATIONS
%----------------------------------------------------------------------------------------

\documentclass[landscape,a0paper,fontscale=0.290]{baposter} % Adjust the font scale/size here

\usepackage{graphicx}     % Required for including images
\graphicspath{{figures/}} % Directory in which figures are stored
\usepackage{listings}

\usepackage{amsmath} % For typesetting math
\usepackage{amssymb} % Adds new symbols to be used in math mode
\usepackage{amsthm, amsfonts, mathtools}
\usepackage{proof}              % For inference rules
\usepackage{mathpartir}         % Automatic rule layout

\usepackage{booktabs} % Top and bottom rules for tables
\usepackage{enumitem} % Used to reduce itemize/enumerate spacing
\usepackage{palatino} % Use the Palatino font
\usepackage[font=small,labelfont=bf]{caption} % Required for specifying captions to tables and figures


\setlength{\columnsep}{1.5em} % Slightly increase the space between columns
\setlength{\columnseprule}{0mm} % No horizontal rule between columns

\usepackage{tikz} % Required for flow chart
\usetikzlibrary{shapes,arrows} % Tikz libraries required for the flow chart in the template

\newcommand{\compresslist}{% Define a command to reduce spacing within itemize/enumerate environments, this is used right after \begin{itemize} or \begin{enumerate}
\setlength{\itemsep}{1pt}
\setlength{\parskip}{0pt}
\setlength{\parsep}{0pt}
}



%----------------------------------------------------------------------------------------
%	MACROS
%----------------------------------------------------------------------------------------

\newcommand\indexVar{k}
\newcommand\lab{lab}

\usepackage{macro/generic}
\usepackage[index=\indexVar,label=\lab]{macro/types}
\usepackage[index=\indexVar,label=\lab]{macro/terms}
\usepackage{macro/inference}
\usepackage{macro/typing}
\usepackage{macro/code}



%----------------------------------------------------------------------------------------
%	POSTER
%----------------------------------------------------------------------------------------

\definecolor{lightblue}{rgb}{0.145,0.6666,1} % Defines the color used for content box headers

\definecolor{borderColor}{rgb}{0.8,0.243,0.243} % Defines the color used for content box headers

\begin{document}

\begin{poster}
{
columns=3,
headerborder=closed, % Adds a border around the header of content boxes
colspacing=1em, % Column spacing
bgColorOne=white, % Background color for the gradient on the left side of the poster
bgColorTwo=white, % Background color for the gradient on the right side of the poster
borderColor=borderColor, % Border color
headerColorOne=black, % Background color for the header in the content boxes (left side)
headerColorTwo=borderColor, % Background color for the header in the content boxes (right side)
headerFontColor=white, % Text color for the header text in the content boxes
boxColorOne=white, % Background color of the content boxes
textborder=roundedleft, % Format of the border around content boxes, can be: none, bars, coils, triangles, rectangle, rounded, roundedsmall, roundedright or faded
eyecatcher=true, % Set to false for ignoring the left logo in the title and move the title left
headerheight=0.1\textheight, % Height of the header
headershape=roundedright, % Specify the rounded corner in the content box headers, can be: rectangle, small-rounded, roundedright, roundedleft or rounded
headerfont=\Large\bf\textsc, % Large, bold and sans serif font in the headers of content boxes
%textfont={\setlength{\parindent}{1.5em}}, % Uncomment for paragraph indentation
linewidth=2pt % Width of the border lines around content boxes
}
%----------------------------------------------------------------------------------------
% TITLE SECTION 
%----------------------------------------------------------------------------------------
%
{\includegraphics[height=4em]{cmu_logo.jpg}} % First university/lab logo on the left
{\bf\textsc{Refinements for Session-typed Concurrency}\vspace{0.5em}} % Poster title
{\textsc{\{ By: Josh Acay \hspace{6pt} Advisor: Frank Pfenning \} \hspace{12pt} Carnegie Mellon University}} % Author names and institution
{\includegraphics[height=4em]{scs_logo.png}} % Second university/lab logo on the right

%----------------------------------------------------------------------------------------
% INTRODUCTION
%----------------------------------------------------------------------------------------

\headerbox{Introduction}{name=introduction,column=0,row=0}{%
%
Session types regulate the communication behaviour along channels between concurrent processes in a typed setting with message-passing concurrency. Recently, a connection between session-types and linear logic have been presented (through the lens of the Curry-Howard correspondence), which gave rise to languages such as SILL \cite{Pfenning15}. Languages incorporating linear session-types enjoy many desirable properties such as global progress, absence of deadlock, and race freedom.

However, vanilla session-types are not strong enough to describe more interesting behavioral properties of concurrent processes. Here, we present a refinement type system in the style of \cite{Freeman91}. Our system combines intersections and unions with equirecursive types, which is strong enough to specify and automatically verify many properties of concurrent programs. We have implemented a type-checker for our system.
}

%----------------------------------------------------------------------------------------
% LANGUAGE
%----------------------------------------------------------------------------------------

\headerbox{Language}{name=language,column=0,below=introduction}{%
%
\setlength\tabcolsep{4pt}

Session types are described by the following grammar:
\begin{center}
\begin{tabular}{l c l l}
  $A, B, C$ & ::= & $\terminate$        & \footnotesize send \texttt{end} and terminate \\
            & $|$ & $A \tensor B$       & \footnotesize send channel of type $A$ and continue as $B$ \\
            & $|$ & $\internals{A}{I}$  & \footnotesize send $\lab_i$ and continue as $A_i$ \\
            & $|$ & $A \lolli B$        & \footnotesize receive channel of type $A$ and continue as $B$ \\
            & $|$ & $\externals{A}{I}$  & \footnotesize receive $\lab_i$ and continue as $A_i$ \\
            %
            & $|$   & $A \intersect B$    & \footnotesize act as both $A$ and $B$ \\
            & $|$   & $A \union B$        & \footnotesize act as either $A$ or $B$
\end{tabular}
\end{center}

Below is a summary of the process expressions, with the sending construct
followed by the matching receiving construct.
\begin{center}
\begin{tabular}{l c l l}
  $P, Q, R$ & ::= & $\tspawn{x}{P_x}{Q_x}$     & \footnotesize cut (spawn) \\
            & $|$ & $\tfwd c d$                & \footnotesize id (forward) \\
            & $|$ & $\tclose c \mid \twait c P$  & \footnotesize $\terminate$ \\
            & $|$ & $\tsend{c}{y}{P_y}{Q} \mid \trecv{x}{c}{R_x}$ & \footnotesize $A \tensor B,$ $A \lolli B$ \\
            & $|$ & $\tselect{c}{}{P} \mid \tcase c {\tbranches Q i}$  \\ && \footnotesize $\externals A I,$ $\internals A I$
\end{tabular}
\end{center}

Note that intersections and unions have no matching constructs since they specify behavioral properties and not structure. Any well-typed program can be given an intersection or a union type. We have three main judgements:
\begin{description}
  \item[$\bracks{A \sub B}$] $A$ is a subtype of $B$
  \item[$\bracks{\typeDJ P c A}$] Process $P$ offers along channel $c$ the session $A$ in the context $\ctx$
  \item[$\bracks{\providesCtx \config \ctx}$] Process configuration $\config$ provides all the channels in $\ctx$
\end{description}
}


%----------------------------------------------------------------------------------------
% INTERSECTIONS AND UNIONS
%----------------------------------------------------------------------------------------

\headerbox{Intersections and Unions}{name=prop,column=1, row=0}{

Typing rules for intersections and unions are given below.

\subsection*{Intersections}
\begin{mathpar}
  \infer[\intersect\Right]{\typeDJ{P}{c}{A \intersect B}}
    {\typeDJ{P}{c}{A} & \typeDJ{P}{c}{B}}
  \and {\infer[\intersect\Left_1]{\typeD{\ctx, c : A \intersect B}{P}{d}{D}}
    {\typeD{\ctx, c : A}{P}{d}{D}}}
  \and {\infer[\intersect\Left_2]{\typeD{\ctx, c : A \intersect B}{P}{d}{D}}
    {\typeD{\ctx, c : B}{P}{d}{D}}}
\end{mathpar}


\subsection*{Unions}
\begin{mathpar}
  {\infer[\union\Right_1]{\typeDJ{P}{c}{A \union B}}
    {\typeDJ{P}{c}{A}}}
  \and {\infer[\union\Right_2]{\typeDJ{P}{c}{A \union B}}
    {\typeDJ{P}{c}{B}}}
  \and \infer[\union\Left]{\typeD{\ctx, c : A \union B}{P}{d}{D}}
    {\typeD{\ctx, c : A}{P}{d}{D} & \typeD{\ctx, c : B}{P}{d}{D}}
\end{mathpar}

}


%----------------------------------------------------------------------------------------
% EXAMPLE
%----------------------------------------------------------------------------------------

\headerbox{Example}{name=example,column=1,below=prop,bottomaligned=language}{%
%
We can define strings of bits using processes. Below, \texttt{Bits} is the type of bit strings. \texttt{emp} represents the empty string, and \texttt{zero} and \texttt{one} append the corresponding bit to the least significant position.

\lstinputlisting[language=krill, style=custom]{code/bitstring.sill}
}


%----------------------------------------------------------------------------------------
% ALGORITHMIC
%----------------------------------------------------------------------------------------

\headerbox{Bidirectional Type Checking}{name=algorithmic,column=2,row=0}{%
%
\setlength{\parskip}{5pt plus 1pt}
Due to the non-structural nature of intersections and unions, there is no obvious way to go from the declarative typing rules to a type-checking algorithm. We therefore designed an algorithmic system, on which the implementation is based.

We write the algorithmic typing judgement as $ \typeA \ctx P c \typeList$ where $\typeList$ is a multiset of types, and $\ctx$ maps a channel to a multiset. On the right, a multiset is interpreted as a disjunction, and on the left, it it interpreted as a conjunction. Additionally, subtyping is handled exclusively at a forward (meaning subsumption cannot be applied freely). Whenever we see an intersection on the left, we break it up into two. We do the same for unions on the right.

The two systems are related through the following soundness and completeness results:
\begin{theorem}[Soundness]
If $\typeA \ctx P c \typeList,$ then $\typeD {\all{\ctx}} P c {\any{\typeList}}$.
\end{theorem}
\begin{theorem}[Completeness]
If $\typeD \ctx P c A,$ then $\typeA \ctx P c A$.
\end{theorem}
}

%----------------------------------------------------------------------------------------
%	RESULTS
%----------------------------------------------------------------------------------------

\headerbox{Metatheory}{name=results,column=2,below=algorithmic}{%
%
We give the standard progress and preservation theorems.

\begin{theorem}[Progress]
If $\providesCtx{\config}{\ctx}$ then either
\begin{enumerate}
  \item $\steps{\config}{\config'}$ for some $\config'$, or
  \item $\config$ is poised.
\end{enumerate}
Intuitively, a process configuration $\config$ is poised if every process in $\config$ is waiting on its client.
\end{theorem}


\begin{theorem}[Preservation]
If $\providesCtx{\config}{\ctx}$ and $\steps{\config}{\config'}$ then $\providesCtx{\config'}{\ctx}$.
\end{theorem}
Proof of preservation is currently in progress.

}


%----------------------------------------------------------------------------------------
% REFERENCES
%----------------------------------------------------------------------------------------

\headerbox{References}{name=references,column=2,below=results, bottomaligned=example}{

\renewcommand{\section}[2]{\vskip 0.05em} % Get rid of the default "References" section title
\nocite{*} % Insert publications even if they are not cited in the poster
\small{ % Reduce the font size in this block
\bibliographystyle{unsrt}
\bibliography{references} % Use references.bib as the bibliography file
}}

%----------------------------------------------------------------------------------------

\end{poster}

\end{document}
