%%%%%%%%%%%%%%%%%%%%%%%%%%%%%%%%%%%%%%%%%
% baposter Landscape Poster
% LaTeX Template
% Version 1.0 (11/06/13)
%
% baposter Class Created by:
% Brian Amberg (baposter@brian-amberg.de)
%
% This template has been downloaded from:
% http://www.LaTeXTemplates.com
%
% License:
% CC BY-NC-SA 3.0 (http://creativecommons.org/licenses/by-nc-sa/3.0/)
%
%%%%%%%%%%%%%%%%%%%%%%%%%%%%%%%%%%%%%%%%%

%----------------------------------------------------------------------------------------
%	PACKAGES AND OTHER DOCUMENT CONFIGURATIONS
%----------------------------------------------------------------------------------------

\documentclass[landscape,a0paper,fontscale=0.285]{baposter} % Adjust the font scale/size here

\usepackage{graphicx} % Required for including images
\graphicspath{{figures/}} % Directory in which figures are stored
\usepackage{verbatim}

\usepackage{amsmath} % For typesetting math
\usepackage{amssymb} % Adds new symbols to be used in math mode
\usepackage{amsthm, amsfonts, mathtools}
\usepackage{proof}              % For inference rules
\usepackage{mathpartir}         % Automatic rule layout

\usepackage{booktabs} % Top and bottom rules for tables
\usepackage{enumitem} % Used to reduce itemize/enumerate spacing
\usepackage{palatino} % Use the Palatino font
\usepackage[font=small,labelfont=bf]{caption} % Required for specifying captions to tables and figures

\setlength{\columnsep}{1.5em} % Slightly increase the space between columns
\setlength{\columnseprule}{0mm} % No horizontal rule between columns

\usepackage{tikz} % Required for flow chart
\usetikzlibrary{shapes,arrows} % Tikz libraries required for the flow chart in the template

\newcommand{\compresslist}{ % Define a command to reduce spacing within itemize/enumerate environments, this is used right after \begin{itemize} or \begin{enumerate}
\setlength{\itemsep}{1pt}
\setlength{\parskip}{0pt}
\setlength{\parsep}{0pt}
}

%----------------------------------------------------------------------------------------
%	MACROS
%----------------------------------------------------------------------------------------

%%%%%%%%%%%%% Theorem environments %%%%%%%%%%%%%
\theoremstyle{plain}
\newtheorem{theorem}{Theorem}
\newtheorem{lemma}[theorem]{Lemma}

%%%%%%%%%%%%%%%%%%%%%%%%%%%%%%%%%%%%%%%%%%%%%%%%


%%%%%%%%%%%%%%%% Generic Macros %%%%%%%%%%%%%%%%

%%% Paired delimiters
\DeclarePairedDelimiter\parens{(}{)}             % parenthesis
\DeclarePairedDelimiter\bracks{\lbrack}{\rbrack} % brackets
\DeclarePairedDelimiter\abs{\lvert}{\rvert}      % absolute value
\DeclarePairedDelimiter\norm{\lVert}{\rVert}     % double verts
\DeclarePairedDelimiter\angled{\langle}{\rangle} % angle brackets
\DeclarePairedDelimiterX\set[2]{\{}{\}}
  {#1 \mathrel{}\mathclose{}\delimsize|\mathopen{}\mathrel{} #2}


%%% Math
\newcommand{\sq}{\text{\ttfamily{\char'15}}} % Single quote
\newcommand{\qq}{\text{\ttfamily"}}          % Double quote
\newcommand{\qquote}[1]{\qq #1\qq{}}         % Strings
%%%%%%%%%%%%%%%%%%%%%%%%%%%%%%%%%%%%%%%%%%%%%%%%


%%%%%%%%%%% Document specific macros %%%%%%%%%%%

%% Types
\newcommand\intersect{\mathbin{\sqcap}}
\newcommand\imp{\supset}
\newcommand\lolli{\multimap}
\newcommand\terminate{\mathbf{1}}


%% Terms
\newcommand\tfwd[2]{#1 \leftarrow #2}
\newcommand\tout[2]{\_ \leftarrow \mathrm{output}\;#1\;#2}
\newcommand\toutc[3]{\_ \leftarrow \mathrm{output}\;#1\;\parens{#2 \leftarrow #3}}
\newcommand\tin[2]{#1 \leftarrow \mathrm{input}\;#2}
\newcommand\tseq[2]{#1 ; #2}
\newcommand\close[1]{\mathrm{close}\;#1}


%% Inference
\newcommand{\irb}[1]{\texttt{#1}}
\newcommand{\intro}{\text{-I}}
\newcommand{\elim}{\text{-E}}
\newcommand\sub[1]{#1\text{-Sub}}

\newcommand{\val}[1]{\ensuremath{{#1}~\mathsf{val}}}

\newcommand{\emptyCtx}{\emptyset}
\newcommand{\ctxFun}{\Gamma}
\newcommand{\ctxChan}{\Psi}
\newcommand{\config}{\Omega}

\newcommand\typeProc[3]{#1 :: \parens{#2 : #3}}
\newcommand\typeS[4]{#1 \vdash \typeProc{#2}{#3}{#4}}
\newcommand\typeSJ{\typeS{\ctxChan}}


\newcommand\stepArrow{\longrightarrow}
\newcommand\steps[2]{#1 \stepArrow #2}
\newcommand\stepsMany[2]{#1 \stepArrow^* #2}
\newcommand\provides[2]{\models #1 :: #2}
\newcommand\proc[2]{\irb{proc}_{#1}\parens{#2}}


%% Induction
\newcommand\pred[1]{\mathcal{P}\parens*{#1}}

%%%%%%%%%%%%%%%%%%%%%%%%%%%%%%%%%%%%%%%%%%%%%%%%


%----------------------------------------------------------------------------------------
%	POSTER
%----------------------------------------------------------------------------------------

\definecolor{lightblue}{rgb}{0.145,0.6666,1} % Defines the color used for content box headers

\definecolor{borderColor}{rgb}{0.8,0.243,0.243} % Defines the color used for content box headers

\begin{document}

\begin{poster}
{
columns=3,
headerborder=closed, % Adds a border around the header of content boxes
colspacing=1em, % Column spacing
bgColorOne=white, % Background color for the gradient on the left side of the poster
bgColorTwo=white, % Background color for the gradient on the right side of the poster
borderColor=borderColor, % Border color
headerColorOne=black, % Background color for the header in the content boxes (left side)
headerColorTwo=borderColor, % Background color for the header in the content boxes (right side)
headerFontColor=white, % Text color for the header text in the content boxes
boxColorOne=white, % Background color of the content boxes
textborder=roundedleft, % Format of the border around content boxes, can be: none, bars, coils, triangles, rectangle, rounded, roundedsmall, roundedright or faded
eyecatcher=true, % Set to false for ignoring the left logo in the title and move the title left
headerheight=0.1\textheight, % Height of the header
headershape=roundedright, % Specify the rounded corner in the content box headers, can be: rectangle, small-rounded, roundedright, roundedleft or rounded
headerfont=\Large\bf\textsc, % Large, bold and sans serif font in the headers of content boxes
%textfont={\setlength{\parindent}{1.5em}}, % Uncomment for paragraph indentation
linewidth=2pt % Width of the border lines around content boxes
}
%----------------------------------------------------------------------------------------
%	TITLE SECTION 
%----------------------------------------------------------------------------------------
%
{\includegraphics[height=4em]{cmu_logo.jpg}} % First university/lab logo on the left
{\bf\textsc{Refinements for Session-typed Concurrency}\vspace{0.5em}} % Poster title
{\textsc{\{ By: Josh Acay \hspace{6pt} Advisor: Frank Pfenning \} \hspace{12pt} Carnegie Mellon University}} % Author names and institution
{\includegraphics[height=4em]{scs_logo.png}} % Second university/lab logo on the right

%----------------------------------------------------------------------------------------
%	INTRODUCTION
%----------------------------------------------------------------------------------------

\headerbox{Introduction}{name=introduction,column=0,row=0}{

The SILL language integrates functional programming and message-passing concurrency. The communication behavior of concurrent message-passing processes are described by session types, which have a foundation in linear logic. SILL's type system provides strong guarantees, such as global progress, absence of deadlock, and race freedom. 

We plan to enhance SILL's type system with refinements to allow the specification and verification of behavioral properties of concurrent programs. Here, we describe an extension of the type system with intersection types, which will serve as the foundations of our refinement system.
}

%----------------------------------------------------------------------------------------
%	SILL EXAMPLE
%----------------------------------------------------------------------------------------

\headerbox{Sill Example}{name=example,column=0,below=introduction}{

We can define strings of bits using processes. Below, \texttt{Bits} is the type of bit strings. \texttt{emp} represents the empty string, and \texttt{zero} and \texttt{one} append the corresponding bit to the least significant position.

\verbatiminput{code/sill.sill}
}

%----------------------------------------------------------------------------------------
%	INTERSECTIONS
%----------------------------------------------------------------------------------------

\headerbox{Intersections}{name=intersections,column=1,row=0}{

Using intersections, we can specify and have the compiler track more properties of functions. For example, if we assume we have a type \texttt{Std} of bit strings with no leading zeros---which is a subtype of \texttt{Bits}---we can augment the type signatures as follows:

\verbatiminput{code/intersection.sill}

The programs themselves do not change.
}

%----------------------------------------------------------------------------------------
%	METATHEORY
%----------------------------------------------------------------------------------------

\headerbox{Metatheory}{name=metatheory,column=1,below=intersections,bottomaligned=example}{

\subsection*{Judgements}

Process $P$ offers along channel $c$ the session $A$:
\[ \typeSJ{P}{c}{A} \]

Process configuration $\config$ provides all the channels in $\ctxChan$:
\[ \provides{\config}{\ctxChan} \]

\subsection*{Intersections}:
\begin{mathpar}
     \infer[\intersect R]{\typeSJ{P}{c}{A_1 \intersect A_2}}
	 { \typeSJ{P}{c}{A_1}
	 & \typeSJ{P}{c}{A_2}
	 }
	\and \infer[\intersect L_1]{\typeS{\ctxChan, c : A_1 \intersect A_2}{P}{c}{B}}
	 { \typeS{\ctxChan, c : A_1}{P}{c}{B}}
	\and \infer[\intersect L_2]{\typeS{\Delta, c : A_1 \intersect A_2}{P}{c}{B}}
	 { \typeS{\ctxChan, c : A_2}{P}{c}{B}}
\end{mathpar}

}

%----------------------------------------------------------------------------------------
%	RESULTS
%----------------------------------------------------------------------------------------

\headerbox{Results}{name=results,column=2,row=0}{

We prove the standard progress and preservation theorems for the language extended with intersection types:

\begin{theorem}[Progress]
If $\provides{\config}{\ctxChan}$ then either
\begin{enumerate}
	\item $\steps{\config}{\config'}$ for some $\config'$, or
	\item $\config$ is poised.
\end{enumerate}
\end{theorem}

\begin{theorem}[Preservation]
If $\provides{\config}{\ctxChan}$ and $\steps{\config}{\config'}$ then $\provides{\config'}{\ctxChan}.$
\end{theorem}

We say that a process configuration $\config$ is poised if every process in $\config$ is waiting on its client. Together, progress and preservation imply type safety.

\begin{theorem}[Type Safety]
Let $\config_0 = \proc{c_0}{P_0}.$ If $\provides{\config_0}{(c_0 : \terminate)}$ and $\stepsMany{\config_0}{\config_n}$ where $\config_n$ cannot make a transition, then $\config_n = \proc{c_0}{\close{c_0}}.$

\end{theorem}
}

%----------------------------------------------------------------------------------------
%	FUTURE WORK
%----------------------------------------------------------------------------------------

\headerbox{Future Work}{name=futurework,column=2,below=results}{

Much work remains to be done. We need a way to specify recursively defined ``sorts'' of user-defined session types. Our type-checker should be able to statically verify the validity of refinements using the subtype relations. Later, we will extend our system to the polarized setting with affine and unrestricted channels.

Once we are done on the theory side, we will implement a prototype type-checker as a proof of concept. We hope to demonstrate the usefulness of our system through examples, and show that our system is able to enhance safety with minimal annotation burden on the programmer. Finally, we might adapt our system to an imperative language with effects and possibly also to the object-oriented setting.
}

%----------------------------------------------------------------------------------------
%	REFERENCES
%----------------------------------------------------------------------------------------

\headerbox{References}{name=references,column=2,below=futurework,bottomaligned=example}{

\renewcommand{\section}[2]{\vskip 0.05em} % Get rid of the default "References" section title
\nocite{*} % Insert publications even if they are not cited in the poster
\small{ % Reduce the font size in this block
\bibliographystyle{unsrt}
\bibliography{references} % Use references.bib as the bibliography file
}}

%----------------------------------------------------------------------------------------

\end{poster}

\end{document}