\documentclass[11pt]{article}

%%%%%%%%%%%%%%%%%%%% Title %%%%%%%%%%%%%%%%%%%%%
\title{Subtyping Session Types}
\date{February 22, 2015}
\author{Josh Acay}
%%%%%%%%%%%%%%%%%%%%%%%%%%%%%%%%%%%%%%%%%%%%%%%%


%%%%%%%%%%%%%% Default packages %%%%%%%%%%%%%%%%
\usepackage{fullpage}           % Page style
\usepackage{amsmath, amsfonts, amsthm, amssymb, mathtools} % Math
\usepackage{proof}              % For inference rules
\usepackage{mathpartir}         % Automatic rule layout
\usepackage{enumerate, calc}    % Environments
%%%%%%%%%%%%%%%%%%%%%%%%%%%%%%%%%%%%%%%%%%%%%%%%


%%%%%%%%%%%%%%%%% Page outline %%%%%%%%%%%%%%%%%
\setlength{\parindent}{0pt}
\setlength{\parskip}{5pt plus 1pt}
%%%%%%%%%%%%%%%%%%%%%%%%%%%%%%%%%%%%%%%%%%%%%%%%


%%%%%%%%%%%%% Theorem environments %%%%%%%%%%%%%
\theoremstyle{plain}
\newtheorem{thm}{Theorem}[section]
\newtheorem{prop}[thm]{Proposition}
\newtheorem{lem}[thm]{Lemma}
\newtheorem{cor}[thm]{Corollary}

\theoremstyle{definition}
\newtheorem{definition}[thm]{Definition}
\newtheorem{example}[thm]{Example}
%\newtheorem{task}{Task}

\theoremstyle{remark}
\newtheorem{remark}[thm]{Remark}

%%%%%%%%%%%%%%%%%%%%%%%%%%%%%%%%%%%%%%%%%%%%%%%%


%%%%%%%%%%%%%%%% Generic Macros %%%%%%%%%%%%%%%%

%%% Paired delimiters
\DeclarePairedDelimiter\parens{(}{)}             % parenthesis
\DeclarePairedDelimiter\bracks{\lbrack}{\rbrack} % brackets
\DeclarePairedDelimiter\abs{\lvert}{\rvert}      % absolute value
\DeclarePairedDelimiter\norm{\lVert}{\rVert}     % double verts
\DeclarePairedDelimiter\angled{\langle}{\rangle} % angle brackets
\DeclarePairedDelimiterX\set[2]{\{}{\}}
  {#1 \mathrel{}\mathclose{}\delimsize|\mathopen{}\mathrel{} #2}


%%% Math
\newcommand{\sq}{\text{\ttfamily{\char'15}}} % Single quote
\newcommand{\qq}{\text{\ttfamily"}}          % Double quote
\newcommand{\qquote}[1]{\qq #1\qq{}}         % Strings
%%%%%%%%%%%%%%%%%%%%%%%%%%%%%%%%%%%%%%%%%%%%%%%%


%%%%%%%%%%% Document specific macros %%%%%%%%%%%

%% Types
\newcommand\imp{\supset}
\newcommand\lolli{\multimap}
\newcommand\intersect{\mathbin{\&\&}}
\newcommand\tout[2]{\_ \leftarrow \mathrm{output}\;#1\;#2}
\newcommand\tin[2]{#1 \leftarrow \mathrm{input}\;#2}
\newcommand\tseq[2]{#1 ; #2}

\newcommand\terminate{\mathbf{1}}


%% Terms


%% Inference
\newcommand{\intro}{\text{-I}}
\newcommand{\elim}{\text{-E}}
\newcommand\sub[1]{#1\text{-Sub}}

\newcommand{\val}[1]{\ensuremath{{#1}~\mathsf{val}}}

\newcommand\typeProc[3]{#1 :: #2 : #3}
\newcommand\typeS[5]{#1; #2 \vdash \typeProc{#3}{#4}{#5}}
\newcommand\typeSJ{\typeS{\Psi}{\Delta}}
\newcommand\exec[3]{\mathrm{exec}\;{#1}\;{#2}\;{#3}}
\newcommand\steps{\longrightarrow}
\newcommand\valid[4]{\models \typeProc{(#1 \mid #2)}{#3}{#4}}


%% Induction
\newcommand\pred[1]{\mathcal{P}\parens*{#1}}

%%%%%%%%%%%%%%%%%%%%%%%%%%%%%%%%%%%%%%%%%%%%%%%%




%%%%%%%%%%%%%%%% Document body %%%%%%%%%%%%%%%%%

\begin{document}

%% Title page
\maketitle

%% Document

\section{Subtyping}
% Positive places are covariant, negative places are contravariant.

\begin{mathpar}
	\infer[\sub{id}]{S \sqsubseteq S}{}
	\and \infer[\sub\wedge]{\tau \wedge S \sqsubseteq \tau' \wedge S'}{\tau \le \tau' & S \sqsubseteq S'}
	\and \infer[\sub\imp]{\tau \imp S \sqsubseteq \tau' \imp S'}{\tau \ge \tau & S \sqsubseteq S'}
	\and \infer[\sub\lolli]{S_1 \lolli S_2 \sqsubseteq S_1' \lolli  S_2'}{S_1' \sqsubseteq S_1 & S_2 \sqsubseteq S_2}
	\and \infer[\sub\otimes]{S_1 \otimes S_2 \sqsubseteq S_1' \otimes S_2'}
		  {S_1 \sqsubseteq S_1' & S_2 \sqsubseteq S_2'}
	\and \infer[\sub\&]{{\&}\{l_i : S_i\}_I \sqsubseteq {\&}\{l_j : S_j'\}_J}
	      {J \subseteq I & S_j \sqsubseteq S_j' \text{ for } j \in J}
	\and \infer[\sub\oplus]{{\oplus}\{l_i : S_i\}_I \sqsubseteq {\oplus}\{l_j : S_j'\}_J}
	      {I \subseteq J & S_i \sqsubseteq S_i' \text{ for } i \in I}
\end{mathpar}

\[
\infer[\text{Sub}]{\typeSJ{P}{c}{S'}}
 { \typeSJ{P}{c}{S}
 & S \sqsubseteq S'
 }
\]

\section{Intersection Types}

\begin{mathpar}
	\infer[\intersect\intro]{\typeSJ{P}{c}{S_1 \intersect S_2}}
	 { \typeSJ{P}{c}{S_1}
	 & \typeSJ{P}{c}{S_2}
	 }
	\\
	\and \infer[\intersect\elim_1]{\typeSJ{P}{c}{S_1}}
	 { \typeSJ{P}{c}{S_1 \intersect S_2}}
	\and \infer[\intersect\elim_2]{\typeSJ{P}{c}{S_2}}
	 { \typeSJ{P}{c}{S_1 \intersect S_2}}
\end{mathpar}


\section{Datatypes}

Are $\otimes$ and $\&$ generative?

\section{Type Preservation}
\begin{thm}
If $\valid{\Sigma}{\Omega}{c_0}{\terminate}$ and $(\Sigma \mid \Omega) \steps (\Sigma' \mid \Omega')$%
\footnote{We carry around $\Sigma$ which is a ''signature of channels``. Shouldn't this be limited to static semantics?} then $\valid{\Sigma'}{\Omega'}{c_0}{\terminate}.$%
\footnote{The theorem in the paper uses $\steps^*.$ Is there a reason one step is not enough?}
\end{thm}
\begin{proof}
Define $\pred{(\Sigma_1 \mid \Omega_1) \steps (\Sigma_2 \mid \Omega_2), \valid{\Sigma_1}{\Omega_1}{c_0}{\terminate}}$ as $\valid{\Sigma_2}{\Omega_2}{c_0}{\terminate}.$ We proceed by lexicographic induction:

\begin{itemize}
	\item Case for $\wedge R$: We have
	\[\Omega_1 = \Omega_1', \exec{(\tseq{\tout{c_m}{V}}{P})}{c_m}{S_m}, \exec{(\tseq{\tin{x}{c_m}}{Q_x})}{c_n}{S_n})\]
	\[ \Omega_2 = \Omega_2', \exec{P}{c_m}{S_m'}, \exec{Q_V)}{c_n}{S_n'}) \]
	
	assuming $\val V.$
	
	By inversion, either $S_m = A \intersect B,$ or $S_m = \tau \wedge A$ where $V : \tau$ and $\typeProc{P}{c_m}{A}.$ In the former case, we case on the derivation of $A \intersect B.$
	
	\begin{itemize}
		\item ()
		\item
	\end{itemize} 
	
\end{itemize}
\end{proof}

\section{Progress}

\end{document}
