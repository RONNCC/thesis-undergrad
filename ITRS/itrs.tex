\documentclass[a4paper,USenglish]{lipics-v2016}
% for section-numbered lemmas etc., use "numberwithinsect"

\usepackage{microtype}%if unwanted, comment out or use option "draft"
\usepackage{proof}
\usepackage{mathpartir}

\usepackage{todonotes}

\newcommand\indexVar{x}
\newcommand\lab{lab}

\usepackage{macro/generic}
\usepackage[index=\indexVar,label=\lab]{macro/language}
\usepackage{macro/code}

\bibliographystyle{plainurl}

% Author macros::begin %%%%%%%%%%%%%%%%%%%%%%%%%%%%%%%%%%%%%%%%%%%%%%%%
\title{Refinements for Session-typed Concurrency\footnote{This work was partially supported by someone.}}
%\titlerunning{Refinements for Session-typed Concurrency}

\author[1]{Co\c{s}ku Acay}
\author[2]{Frank Pfenning}
\affil[1]{Carnegie Mellon University, Pittsburgh, PA 15213 \\
  \texttt{cacay@cmu.edu}}
\affil[2]{Carnegie Mellon University, Pittsburgh, PA 15213 \\
  \texttt{fp@cs.cmu.edu}}
\authorrunning{C. Acay and F. Pfenning} %mandatory. First: Use abbreviated first/middle names. Second (only in severe cases): Use first author plus 'et. al.'

\Copyright{Cosku Acay}%mandatory, please use full first names. LIPIcs license is "CC-BY";  http://creativecommons.org/licenses/by/3.0/

\subjclass{D.3.2 Language Classifications}% mandatory: Please choose ACM 1998 classifications from http://www.acm.org/about/class/ccs98-html . E.g., cite as "F.1.1 Models of Computation". 
\keywords{Dummy keyword -- please provide 1--5 keywords}% mandatory: Please provide 1-5 keywords
% Author macros::end %%%%%%%%%%%%%%%%%%%%%%%%%%%%%%%%%%%%%%%%%%%%%%%%%

%Editor-only macros:: begin (do not touch as author)%%%%%%%%%%%%%%%%%%%%%%%%%%%%%%%%%%
\EventEditors{John Q. Open and Joan R. Acces}
\EventNoEds{2}
\EventLongTitle{42nd Conference on Very Important Topics (CVIT 2016)}
\EventShortTitle{CVIT 2016}
\EventAcronym{CVIT}
\EventYear{2016}
\EventDate{December 24--27, 2016}
\EventLocation{Little Whinging, United Kingdom}
\EventLogo{}
\SeriesVolume{42}
\ArticleNo{23}
% Editor-only macros::end %%%%%%%%%%%%%%%%%%%%%%%%%%%%%%%%%%%%%%%%%%%%%%%

\begin{document}

\maketitle

\begin{abstract}
Prior work has explored the deep connection between linear sequent calculus and session-typed message-passing concurrent computation by combining equirecursive types with a natural notion of subtyping. In this paper, we extend this further by intersection and union types in order to express multiple behavioral properties of processes in a single type. In the resulting system, we can represent internal and external choice by intersection and union, as first suggested by Padovani for a different system of session types.
\todo{Better abstract.}
\end{abstract}

 \todo{Find best subjclass}
 \todo{Pick keywords}
 \todo{Are we sponsored by someone?}

\section{Typesetting instructions -- please read carefully}
Please comply with the following instructions when preparing your article for a LIPIcs proceedings volume. 
\begin{itemize}
\item Fill out the \verb+\subjclass+ and \verb+\keywords+ macros. For the \verb+\subjclass+, please refer to the ACM classification at \url{http://www.acm.org/about/class/ccs98-html}.
\item Take care of suitable linebreaks and pagebreaks. No overfull \verb+\hboxes+ should occur in the warnings log.
\end{itemize}


\section{Introduction}

Prior work has established a Curry-Howard correspondence between intuitionistic linear sequent calculus and session-typed message-passing concurrency \cite{CairesP10, PfenningG15, Honda93}. In this formulation, linear propositions are interpreted as session-types, proofs as processes, and cut elimination as communication. Session types are assigned to channels and prescribe the communication behaviour along them. Each channel is offered by a unique process and used by exactly one, which is where linearity comes in. When the behaviour along a channel $c$ satisfies the type $A$ and $P$ is the process that offers along $c$, we say that $P$ provides along $c$ the type $A$.

In the base system, each type directly corresponds to process of a certain form. For example, a process providing the type $A \tensor B$ first sends out a channel satisfying $A$, then acts as $B$. Similarly, a process offering $\terminate$ sends the label $\irb{end}$ and terminates. We call these \emph{structural} types since they correspond to processes of a certain structure. In this paper, we extend the base type system with intersections and unions. We call these \emph{property} types since they do not correspond to specific forms of processes in that any process may be assigned such a type. In addition, if we interpret a type as specifying a property, then intersection corresponds to satisfying two properties simultaneously and union corresponds to satisfying one or the other. When types are interpreted as sets, intersection and union are set theoretic intersection and union. \todo{I don't think I need the set theoretic interpretation, but there are many ways to think about intersections and unions. Which ones are important?}

Our goal is to show that the base system extended with intersection, unions, recursive types, and a natural notion of subtyping is type safe. We do this by proving the usual type preservation and progress theorems, which correspond to session fidelity and deadlock freedom in the concurrent context. In the presence of a strong subtyping relation and transparent (i.e.\ non-generative)  equirecursive types, intersections and unions turn out to be powerful enough to specify many interesting communications behaviours, which we demonstrate with concrete examples and by analogy to \cite{FreemanP91}.

Our contributions are summarized below:
\begin{itemize}
  \item We introduce intersection and union types to a session-typed concurrent calculus and prove standard type preservation and progress theorems.
  \item We give a simple and sound coinductive subtyping relation in the presence of equirecursive types, intersections, and unions reminiscent of Gentzen's multiple conclusion logic \cite{Gentzen35, Girard87}.
  \item We show how intersections and unions can be used as refinements of recursive types in the style of \cite{FreemanP91} in a linear setting.
  \item We show that subtyping and type checking are decidable by presenting an algorithmic system.
  \item We demonstrate how certain structural type constructs (i.e.\ internal and external choices) can be understood as singletons interacting with intersection and union.
\end{itemize}

\todo{Non contributions.}
Some aspects that would be important in a full accounting of the system but are left out for the sake of conciseness are as follows:
\begin{itemize}
  \item We omit discussion of an underlying functional language and only consider the process calculus. We believe the integration is orthogonal and can be found elsewhere \cite{ToninhoCP13}.
  \item We do not consider shared/unrestricted channels and believe they are a simple extension \cite{CairesP10}.
  \item Asynchronous communication does not depend on the specifics of the type system and should be trivial \cite{CairesP10}.
  \item Polymorphism and abstract types are not in the system. Depending on the specifics, polymorphism can be tricky in the presence of equirecursive types and requires some care. \todo{Citation.}
\end{itemize}

The rest of this paper is structured as follows. In section~\ref{base}, we formally introduce the base system and explore the correspondence between linear logic and session types more. Section~\ref{recursive} extends the base system with recursive types and a natural notion of coinductive subtyping. Section~\ref{refinements} is where our contribution starts. We add intersections and unions and modify the subtyping relation to handle distributivity cleanly. Section~\ref{metatheory} discusses the metatheory, section~\ref{algorithmic} gives an algorithmic system for type-checking, and finally, section~\ref{conclusion} concludes.


\section{From Linear Logic to Session Types}
\label{base}
We only give a brief review of linear logic and its connection to session types here. Interested readers are referred to \cite{CairesP10, PfenningG15, Honda93}. The key idea of linear logic is to treat logical propositions as resources: each must be used \emph{exactly} once. In the intuitionistic setting, this is written as a linear hypothetical judgement:
$$ A_1, \ldots, A_n \vdash A $$
where $A_1$ through $A_n$ are the hypothesis that must be used exactly once in the proof of $A$. The order of hypothesis is irrelevant, so they are treated as a (multi) set. According to the Curry-Howard isomorphism for intuitionistic linear logic, propositions are interpreted as session types, proofs as concurrent processes, and cut elimination as communication. For this correspondence, hypothesis are labelled with channels (rather than with variables). We also assign a channel name to the conclusion since processes are not evaluated like in a functional language but are communicated with (along a channel). This leads to the following judgement:
$$ \typeD {c_1 : A_1, \ldots, c_n : A_n} P c A$$
which should be interpreted as ``$P$ offers along the channel $c$ the session $A$ using channels $c_1, \ldots, c_n$ with the corresponding types''. We assume $c_1, \ldots, c_n$ and $c$ are all distinct.

Each process offers along a specific channel, and in the linear setting, each channel must be used exactly in one place. Processes cannot rename channels, which suggests that channel names act as unique process identifiers.

Within this framework, the $\cut$ rule corresponds to a form of process composition where the main process (client) spawns off a helper process (provider) which it can communicate with from then on:
$$ \infer[\cut]{ \typeD {\ctx, \ctx'} {\tspawn{c}{P_c}{Q_c}} {d} {D} }
    { \typeD {\ctx} {P_c} {c} {A}
    & \typeD {\ctx', c : A} {Q_c} {d} {D}
    }
$$
\todo{Mention the system is different from $\pi$ calculus?}.

Working out the isomorphism further and assigning a session type to each linear proposition gives the following interpretation:

\begin{center}
\begin{tabular}{l c l l}
  $A, B, C$ & ::= & $\terminate$        & send \texttt{end} and terminate \\
            & $|$ & $A \tensor B$       & send channel of type $A$ and continue as $B$ \\
            & $|$ & $\internals{A}{I}$  & send $\lab_i$ and continue as $A_i$ \\
            & $|$ & $\tau \sendVal B$   & send value of type $\tau$ and continue as $B$ \\
            & $|$ & $A \lolli B$        & receive channel of type $A$ and continue as $B$ \\
            & $|$ & $\externals{A}{I}$  & receive $\lab_i$ and continue as $A_i$ \\
            & $|$ & $\tau \recvVal B$   & receive value of type $\tau$ and continue as $B$
\end{tabular}
\end{center}

The process with trivial behaviour is typed by $\terminate$. $A \tensor B$ and $A \lolli B$ correspond to sending and receiving channel names respectively. $\tau \sendVal B$ and $\tau \recvVal B$ are similar, but values in some underlying functional language are exchanged rather than channels. We will ignore these types for the sake of conciseness and focus on the process calculus. The integration of a functional language is orthogonal and can be found in \cite{ToninhoCP13}.

Possibly the more interesting types are $\internals A I$ and $\externals A I$ which are generalizations of the binary additive disjunction ($\internal$) and conjunction ($\external$). $\internals A I$ is called an internal choice, since the label is picked by the provider (we always consider the world from the provider's perspective). In the same vein, $\externals A I$ is an external choice since the choice is made externally by the client. In either case, $I$ is a \emph{finite} index set, $\lab : I \to \labels$ is an \emph{injective} function into the set of labels, and $A : I \to \types$ is any function into types. The order of labels does not matter and each label must be unique.


\subsection{Process Expressions}

The proof terms, or processes, corresponding to these types are given below with the sending construct followed by the receiving construct. Depending on whether the communication happens along the provided channel or one of the used channels, we get either a type or its dual. \todo{This sentence sounds wack}.

\begin{center}
\begin{tabular}{l c l l}
  $P, Q, R$ & ::= & $\tspawn{x}{P_x}{Q_x}$     & cut (spawn) \\
            & $|$ & $\tfwd c d$                & id (forward) \\
            & $|$ & $\tclose c \mid \twait c P$  & $\terminate$ \\
            & $|$ & $\tsend{c}{y}{P_y}{Q} \mid \trecv{x}{c}{R_x}$ & $A \tensor B,$ $A \lolli B$ \\
            & $|$ & $\tselect{c}{}{P} \mid \tcase c {\tbranches Q i}$  & $\externals A I,$ $\internals A I$ \\
            & $|$ & $\tsendVal{c}{M}{Q} \mid \trecvVal{n}{c}{R_n}$ & $A \sendVal B,$ $A \recvVal B$
\end{tabular}
\end{center}

An example program will give more intuition about the system. We will look at process level natural numbers, which will also be our running example in this paper. Note that we will use concrete syntax, but the mapping to abstract syntax presented above should be clear. Also, this example (and almost any interesting one) requires recursive types, which are introduced in the next section. However, it should be good enough to give some intuition.

First, we define the interface:

\begin{lstlisting}[language=krill, style=custom]
  type Nat = +{zero : 1, succ : Nat}
\end{lstlisting}

This states a process level natural number is an internal choice of either zero or a successor of another natural. Next, we define two simple processes that implement the interface:

\begin{lstlisting}[language=krill, style=custom]
  zero : Nat
  `c <- zero = do
    `c.zero
    close `c

  succ : Nat -o Nat
  `c <- succ `d = do
    `c.succ
    `c <- `d
\end{lstlisting}

\texttt{zero} simply sends the label \texttt{zero} and terminates, whereas \texttt{suc} appends a successor to the given process. Here is a slightly more complicated example that doubles the given natural:

\begin{lstlisting}[language=krill, style=custom]
  double : Nat -o Nat
  `c <- double `d =
    case `d of
      zero -> do wait `d; `c.zero; close `c
      succ -> do
        `c.succ
        `c.succ
        `c <- double `d
\end{lstlisting}

These are very simple examples, though we hope they offer some insight into the system. We will expand on these example when we introduce intersections and unions, and give more interesting ones at the end once we have the whole system.


\subsection{Type Assignment for Processes}

The typing rules for other constructs are derived from linear logic by decorating derivations with proof terms just as we did with $\cut$. The rules are given in Figure~\ref{session-assignment}. \todo{Talk more about these maybe?}

\begin{rules}[session-assignment]{Type assignment for process expressions}
  % id and cut
  \infer[\id]{ \typeD {c : A} {\tfwd{d}{c}} {d} {A} }
    {}
  \and \infer[\cut]{ \typeD {\ctx, \ctx'} {\tspawn{c}{P_c}{Q_c}} {d} {D} }
    { \typeD {\ctx} {P_c} {c} {A}
    & \typeD {\ctx', c : A} {Q_c} {d} {D}
    }
  % Terminate
  \and \infer[\terminate\Right]{\typeD{\emptyCtx}{\tclose c}{c}{\terminate}}
    {}
  \and \infer[\terminate\Left]{\typeD{\ctx, c : \terminate}{\twait c P}{d}{A}}
    {\typeDJ{P}{d}{A}}
  % Tensor
  \and \infer[\tensor\Right]{\typeD{\ctx, \ctx'}{\tsend{c}{d}{P_d}{Q}}{c}{A \tensor B}}
    { \typeD{\ctx}{P}{d}{A}
    & \typeD{\ctx'}{Q}{c}{B}
    }
  \and \infer[\tensor\Left]{ \typeD{\ctx, c : A \tensor B}{\trecv{d}{c}{P_d}}{e}{E} }
    { \typeD{\ctx, d : A, c : B}{P_d}{e}{E} }
  % Internal choice
  \and \infer[\internal\Right]{\typeDJ { \tselect{c}{i}{P} } {c} {\internals{A}{I}} }
    { i \in I
    & \typeDJ{P}{c}{A_i}
    }
  \and \infer[\internal\Left]{ \typeD { \ctx, c : \internals{A}{I} } { \tcase{c}{\tbranches{P}{J}} } {d} {D} }
   { I \subseteq J
   & \typeD{\ctx, c : A_k}{P_k}{d}{D}~\text{for}~k\in I
   }
  % Implication
  \and \infer[\lolli\Right]{ \typeD{\ctx}{\trecv{d}{c}{P_d}}{c}{A \lolli B} }
    { \typeD{\ctx, d : A}{P_d}{c}{B} }
  \and \infer[\lolli\Left]{\typeD{\ctx, \ctx', c : A \lolli B}{ \tsend{c}{d}{P_d}{Q} } {e}{E}}
    { \typeD{\ctx}{P_d}{d}{A}
    & \typeD{\ctx', c : B}{Q}{e}{E}
    }
  % External choice
  \and \infer[\external\Right]{ \typeDJ { \tcase{c}{\tbranches{P}{I}} } {c} {\externals{A}{J}} }
   { J \subseteq I
   & \typeDJ{P_k}{c}{A_k}~\text{for}~k\in J
   }
  \and \infer[\external\Left]{\typeD{\ctx, c : \externals{A}{I}} { \tselect{c}{i}{P} } {d} {D}}
    { i \in I
    & \typeD{\ctx, c : A_i}{P}{d}{D}
    }
\end{rules}


\section{Recursive Types and Subtyping}
\label{recursive}


\section{Intersections and Unions}
\label{refinements}

  \subsection{Intersection Types}

  \subsection{Union Types}

  \subsection{Subtyping}


\section{Metatheory}
\label{metatheory}

\begin{theorem}[Progress]
If $\providesCtx \config \ctx$ then either
\begin{enumerate}
  \item $\steps{\config}{\config'}$ for some $\config'$, or
  \item $\config$ is poised.
\end{enumerate}
\end{theorem}


\begin{theorem}[Preservation]
If $\providesCtx \config \ctx$ and $\steps{\config}{\config'}$ then $\providesCtx {\config'} \ctx$.
\end{theorem}


\section{Algorithmic System}
\label{algorithmic}


\section{Conclusion}
\label{conclusion}


\appendix


%%
%% Bibliography
%%


\bibliography{bibliography}



\end{document}
